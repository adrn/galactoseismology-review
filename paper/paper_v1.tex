\documentclass[galaxies,article,submit,moreauthors,pdftex,10pt,a4paper]{mdpi} 
\bibliographystyle{newapa}

%--------------------
% Class Options:
%--------------------
% journal
%----------
% Choose between the following MDPI journals:
% actuators, admsci, aerospace, agriculture, agronomy, algorithms, animals, antibiotics, antibodies, antioxidants, applsci, arts, atmosphere, atoms, axioms, batteries, bdcc, behavsci, beverages, bioengineering, biology, biomedicines, biomimetics, biomolecules, biosensors, brainsci, buildings, carbon, cancers, catalysts, cells, challenges, chemengineering, chemosensors, children, chromatography, climate, coatings, computation, computers, condensedmatter, cosmetics, cryptography, crystals, data, dentistry, designs, diagnostics, diseases, diversity, econometrics, economies, education, electronics, energies, entropy, environments, epigenomes, fermentation, fibers, fishes, fluids, foods, forests, fractalfract, futureinternet, galaxies, games, gastrointestdisord, gels, genealogy, genes, geosciences, geriatrics, healthcare, horticulturae, humanities, hydrology, informatics, information, infrastructures, inorganics, insects, instruments, ijerph, ijfs, ijms, ijgi, ijtpp, inventions, jcdd, jcm, jcs, jdb, jfb, jfmk, jimaging, jof, jintelligence, jlpea, jmmp, jmse, jpm, jrfm, jsan, land, languages, laws, life, literature, logistics, lubricants, machines, magnetochemistry, marinedrugs, materials, mathematics, mca, mti, medsci, medicines, membranes, metabolites, metals, microarrays, micromachines, microorganisms, minerals, molbank, molecules, mps, nanomaterials, ncrna, neonatalscreening, nitrogen, nutrients, ohbm, particles, pathogens, pharmaceuticals, pharmaceutics, pharmacy, philosophies, photonics, plants, polymers, proceedings, processes, proteomes, publications, quaternary, qubs, recycling, religions, remotesensing, resources, risks, robotics, safety, scipharm, sensors, separations, sexes, sinusitis, socsci, societies, soilprocesses, soils, sports, standards, sustainability, symmetry, systems, technologies, toxics, toxins, tropicalmed, universe, urbansci, vaccines, vetsci, viruses, vision, water
%---------
% article
%---------
% The default type of manuscript is article, but can be replaced by: 
% addendum, article, benchmark, book, bookreview, briefreport, casereport, changes, comment, commentary, communication, conceptpaper, correction, conferenceproceedings, conferencereport, expressionofconcern, meetingreport, creative, datadescriptor, discussion, editorial, essay, erratum, hypothesis, interestingimage, letter, newbookreceived, opinion, obituary, projectreport, reply, reprint, retraction, review, perspective, preprints, protocol, shortnote, supfile, technicalnote, viewpoint
% supfile = supplementary materials
%----------
% submit
%----------
% The class option "submit" will be changed to "accept" by the Editorial Office when the paper is accepted. This will only make changes to the frontpage (e.g. the logo of the journal will get visible), the headings, and the copyright information. Also, line numbering will be removed. Journal info and pagination for accepted papers will also be assigned by the Editorial Office.
%------------------
% moreauthors
%------------------
% If there is only one author the class option oneauthor should be used. Otherwise use the class option moreauthors.
%---------
% pdftex
%---------
% The option pdftex is for use with pdfLaTeX. If eps figure are used, remove the option pdftex and use LaTeX and dvi2pdf.

%=================================================================
\firstpage{1} 
\makeatletter 
\setcounter{page}{\@firstpage} 
\makeatother 
\articlenumber{x}
\doinum{10.3390/------}
\pubvolume{xx}
\pubyear{2017}
\copyrightyear{2017}
\externaleditor{Academic Editor: name}
\history{Received: date; Accepted: date; Published: date}

%------------------------------------------------------------------
% The following line should be uncommented if the LaTeX file is uploaded to arXiv.org
%\pdfoutput=1

%=================================================================
% Add packages and commands here. The following packages are loaded in our class file: fontenc, calc, indentfirst, fancyhdr, graphicx, lastpage, ifthen, lineno, float, amsmath, setspace, enumitem, mathpazo, booktabs, titlesec, etoolbox, amsthm, hyphenat, natbib, hyperref, footmisc, geometry, caption, url, mdframed, tabto, soul, multirow, microtype

%=================================================================
%% Please use the following mathematics environments: Theorem, Lemma, Corollary, Proposition, Characterization, Property, Problem, Example, ExamplesandDefinitions, Hypothesis, Remark, Definition
%% For proofs, please use the proof environment (the amsthm package is loaded by the MDPI class).

%=================================================================
% Full title of the paper (Capitalized)
\Title{Disk Heating, Galactoseismology and the Formation of Stellar Halos}

% If this is an expanded version of a conference paper, please cite it here: enter the full citation of your conference paper, and add $^\dagger$ in the end of the title of this article.
%\conference{Title}

% Authors, for the paper (add full first names)
\Author{Kathryn Johnston $^{1,\dagger,\ddagger}$, Firstname Lastname $^{1,\ddagger}$ and Sanjib Sharma $^{2,}$*}
% Rachel Beaton? Steve Majewski? Sanjib Sharma?

% Authors, for metadata in PDF
\AuthorNames{Firstname Lastname, Firstname Lastname and Firstname Lastname}

% Affiliations / Addresses (Add [1] after \address if there is only one affiliation.)
\address{%
$^{1}$ \quad Affiliation 1; e-mail@e-mail.com\\
$^{2}$ \quad Affiliation 2; e-mail@e-mail.com}

% Contact information of the corresponding author
\corres{Correspondence: e-mail@e-mail.com; Tel.: +x-xxx-xxx-xxxx}

% Current address and/or shared authorship
\firstnote{Current address: Affiliation 3} 
\secondnote{These authors contributed equally to this work.}
% The commands \thirdnote{} till \eighthnote{} are available for further notes

% Simple summary
%\simplesumm{}

% Abstract (Do not use inserted blank lines, i.e. \\) 
\abstract{
%A single paragraph of about 200 words maximum. 1) Background: Place the question addressed in a broad context and highlight the purpose of the study; 2) Methods: Describe briefly the main methods or treatments applied; 3) Results: Summarize the article's main findings; and 4) Conclusion: Indicate the main conclusions or interpretations. The abstract should be an objective representation of the article, it must not contain results which are not presented and substantiated in the main text and should not exaggerate the main conclusions.
Deep star count surveys around the Milky Way have revealed diffuse structures encircling our Galaxy, at distances from the Galactic center and heights above and below the Galactic plane far beyond where the stellar disk has historically been thought to end.
This paper reviews results from our own and other observational programs, which together suggest that, despite their extreme positions, the stars in these structures were nevertheless formed in our disk.
Mounting evidence from observations and simulations implies connections between several of these distinct structures, indicative of collective disk oscillations that can plausibly be traced all the way to asymmetries seen in the stellar velocity distribution around the Sun.
There are multiple interesting implications of these findings: 
they promise new perspectives on the process of disk heating;
they give direct evidence for one avenue for the formation of the stellar halo in addition to accretion of satellite galaxies; 
and they motivate the mining of current and near future surveys to trace these oscillations across the Galaxy on global scales.
Such maps could be used as dynamical diagnostics in the  emerging field of ''Galactoseismology'' which promises to probe ongoing interactions between the Milky Way and its entourage of satellites, as well examine the density of our dark matter halo.
Moreover, as sensitivity to very low surface brightness features around external galaxies increases, it is likely that many more examples of such disk oscillations will be revealed, and the detailed understanding of Galactoseismology developed for the Milky Way could be applied as an additional avenue to  constrain galactic interaction rates and dark matter halo densities for samples that are statistically significant on cosmic scales.}

% Keywords
\keyword{galaxy formation; galactic disks; stellar halos; Milky Way.}

% The fields PACS, MSC, and JEL may be left empty or commented out if not applicable
%\PACS{J0101}
%\MSC{}
%\JEL{}

%%%%%%%%%%%%%%%%%%%%%%%%%%%%%%%%%%%%%%%%%%
% Only for journal Applied Sciences:
%\featuredapplication{Authors are encouraged to provide a concise description of the specific application or a potential application of the work. This section is not mandatory.}
%%%%%%%%%%%%%%%%%%%%%%%%%%%%%%%%%%%%%%%%%%


%%%%%%%%%%%%%%%%%%%%%%%%%%%%%%%%%%%%%%%%%%
% Only for the journal Data:
%\dataset{DOI number or link to the deposited data set in cases where the data set is published or set to be published separately. If the data set is submitted and will be published as a supplement to this paper in the journal Data, this field will be filled by the editors of the journal. In this case, please make sure to submit the data set as a supplement when entering your manuscript into our manuscript editorial system.}

%\datasetlicense{license under which the data set is made available (CC0, CC-BY, CC-BY-SA, CC-BY-NC, etc.)}

%%%%%%%%%%%%%%%%%%%%%%%%%%%%%%%%%%%%%%%%%%
% For Conference Proceedings Papers:
%\conferencetitle{Add the conference title here}

%\setcounter{secnumdepth}{4}
%%%%%%%%%%%%%%%%%%%%%%%%%%%%%%%%%%%%%%%%%%

\begin{document}

%%%%%%%%%%%%%%%%%%%%%%%%%%%%%%%%%%%%%%%%%%

%%%%%%%%%%%%%%%%%%%%%%%%%%%%%%%%%%%%%%%%%%
%\setcounter{section}{-1} %% Remove this when starting to work on the template.

\section{Introduction}

Our perspective on the Milky Way presents both unique challenges and unique opportunities  in contributing to our quest to understand galaxies more generally.
Since we live inside it, it is the one galaxy in the Universe that we cannot take an image of but rather need to survey the entire sky in order to map its structure.
On the other hand, it is one of the few galaxies that we can study star-by-star and is the only one we can hope to
%move beyond co-ordinates restricted to random projections in a limited number of directions to 
make a map of in both the full three dimensions of space and the full three dimensions of motion. 

We are in the middle of a renaissance in Milky Way studies, fueled by stellar data sets of sufficient scope in terms of sky coverage and numbers to exploit our perspective as a strength rather than a liability 
%and take full advantage of our proximity 
to create vast catalogues of stellar data.
The catalogues that inspired this renaissance emerged in the 1990's.
They not only mapped global structures in our Galaxy, but also revealed the ubiquity of substructure within it, adding an unforseen richness to interpretations, and new dynamical tools for probing ongoing interactions and formation histories:
HIPPARCOS \citep{esa97} found moving groups of stars in the velocity distribution in the Solar Neighborhood \citep{dehnen98}, some corresponding (as expected) to destroyed stars clusters, while others (unexpectedly) could plausibly be explained as signatures of resonances with the Galactic bar \citep{dehnen00};
the Sloan Digital Sky Survey \citep[hereafter, SDSS ---][]{york00,stoughton02,abazajian03} found many streams of Main Sequence Turnoff (MSTO) stars in our stellar halo, thought to be from long-dead satellite galaxies \citep{newberg02,belokurov06} as a stunning confirmation that our Galaxy had indeed formed hierarchically \citep[e.g.][]{bullock01,bullock05};
and M giant stars selected from the Two Micron All Sky Survey \citep[hereafter, 2MASS ---][]{nikolaev00} traced the debris from the Sagittarius dwarf galaxy entirely around the Sky \citep{majewski03} to offer a window on the history of its disruption \citep{law10}.

\begin{figure}[t]
\label{fig:ting}
\centering
%\includegraphics[width=3 cm]{logo-mdpi}
\caption{\label{fig:ting}
Summary of spatial distributions of M giants}
\end{figure}   

Figure \ref{fig:ting}, reproduced from \cite{li17}, shows the spatial distribution of M giant stars associated with three such substructures that this paper concentrates on. Note that this map was made using estimates from photometry  for the distances to these stars from the Sun.
The expected distance uncertainties using this method are of order 20\% \citep{sheffield14}, so any fine features discussed below will not be apparent.
These substructures were found in SDSS and 2MASS as deviations from the expected global structure of the disk or inner stellar halo around the anti-center and at a range of low to moderate Galactic latitudes --- we will hereafter refer to them collectively as the ``low-latitude structures''.
Note that the same region on the sky has been shown to be richly structured on even smaller  scales\citep{slater14,martin14,deason14} but we do not discuss this sub-substructure here.
\begin{description}
\item{\it The Monoceros Ring or Giant Anticenter Stellar Structure(hereafter Mon/GASS)} is a distinct ring of stars beyond the expected edge of the Galactic disk \citep[at $\sim$ 5 kpc beyound the Sun][]{robin92},  spanning the longitude range ???, with Galactic latitudes ??? and distances from the Sun $\sim$?? kpc. It was originally found as excess of main sequence stars in SDSS \citep{yanny03,ibata03} and M giant stars in 2MASS \citep{rochapinto03}. Follow-up spectra of the M giant stars indicate that they follow a clear trend in mean velocity with Galactic longitude with small dispersion \citep{crane04}.
\item{\it The Triangulum-Andromeda Clouds, I and II,} were first discovered as a single diffuse over-density of M giant stars, covering the area ????, overlapping the Mon/GASS structure on the sky, but at distances of approximately 15-20 kpc from the Sun \citep{rochapinto04}. 
Spectra showed that the M giants exhibited a coherent velocity structure with small dispersion  \citep{rochapinto04} and  deep photometry in the region allowed the clear identification of MSTO stars and an estimate of a surface brightness $\Sigma >$ 32 mag/arcsec$^2$ \cite{majewski04}.
Subsequent photometric work indicated the presence of {\it two} distinct MSTO \citep{martic07}.
\item{\it A13} is another tenuous association of stars discovered by applying a group finding algorithm \citep{sharma09} to the catalogue of all 2MASS M giants \citep{sharma10}. 
This group sits in the north Glalactic hemisphere in the area ???? on the sky and at approximate distances of ???.
\end{description}
There are many models for the formation of these substructures.
Mon/GASS has attributed to the accretion of a satellite \citep[on a retrograde orbit][]{penarrubia05, michel-dansac11}, a natural extension of the Galactic warp \citep{momany04,momany06} or disturbances to the Galactic disk \citep{kazantzidis08,younger08,purcell11,gomez16}.
For TriAnd I/II and A13, their extreme position at (R,Z)$\sim$(kpc,kpc) seems to exclude a disturbed disk as a possible origin in these cases, and a satellite on a retrograde orbit has again been shown to provide a plausible explanation \citep{sheffield14}.

A more convincing, coherent picture of the nature of these low-latitude structures is just beginning to emerge. 
This paper reviews the contributions our own group has made to formulating this picture, which include surveys of the low-latitude structures in spectra \citep{sheffield14,ting17}, stellar populations \citep{pricewhelan15,sheffield17} and abundance patterns \citep{bergemann17}, as well as numerical simulations \citep{sheffield14,laporte17a,laporte17b}.
We summarize this observational and theoretical work in Section 2 and 3 respectively, adding in the context of contemporary work from other groups, as well as the larger context of possible connections across the Galactic disk.
Armed with this understanding of the nature of these substructures, we go on in Section 4 to discuss prospects for mapping such structures more generally around our own and other Galaxies.
We end in Section 5 by outlining the motivation for making such maps, asking what they might be telling us about bigger questions: galaxy formation scenarios and dark matter distributions.


\section{The Nature of Structures Around the Outer Disk --- Summary of Observations}
\label{sec:obs}
\subsection{Spectroscopic surveys}

Over the last five years our group has pursued a spectroscopic campaign to follow up 2MASS M giants that had been identified as groups in position and photometry, with the aims of confirming that these stars were indeed associated, mapping their velocity structures and producing models of their formation. 
We were interested in particular in {\it avoiding} the collimated stellar streams that had been well-studied in prior work \citep[such as Sgr, Orphan, GD1 and Pla 5 --- see, e.g.,][]{law10,koposov10,kuepper15,bovy16}  
and instead looking at groups that appeared as diffuse, amorphous, extended stellar structures such as the TriAnd Clouds \citep{rochapinto04} and A13 \citep{sharma10}.
These morphologies suggested these structures could be  {\it shells} (seen from an internal perspective) --- debris from the disruption of satellite galaxies on radial orbits \citep{johnston08}.

Our first results, extending the \cite{rochapinto04} velocity sample of the TriAnd region to all stars identified as members in \cite{sharma10}, were published in  \cite{sheffield14}.
We found that the groups members formed 2 sequences in color-magnitude space, which we identified with the two distinct MSTO in optical data by \cite{martin07}, and named TriAnd I and TriAnd II respectively.
Both sequences formed a clear over-density in velocity, with a low velocity dispersion ($\sim 25$ km/s)  and steady gradient of average velocity with $l$.
We successfully modeled TriAnd I/II simultaneously as debris stripped over two separate pericentric passages from a single accreted satellite on a low-eccentricity, retrograde, near planar orbit. 
However, these debris structures were morphologically closer to {\it streams} rather than {\it shells}.

\cite{li17} continued the spectroscopic survey to look at A13, again finding coherent velocity structures with low dispersion. 

\begin{figure}[t]
\centering
%\includegraphics[width=3 cm]{logo-mdpi}
\caption{\label{fig:ting_vel}
Summary of velocity distributions of M giants}
\end{figure}   


Figure \ref{fig:ting_vel}, reproduced from \cite{li17}, summarizes the M giant line-of-sight velocity results for all our low-latitude structures. 
Not only do these groups all have low dispersion, confirming the association of stars within each group, they also appear to collectively exhibit a continuous, gentle gradient with $l$.
This latter finding is suggestive of the groups being associated with each other, forming a group of groups. 

%{\it Summary: the low latitude structures each have low velocity dispersions supporting the genuine association of the stars within them; they also form velocity sequences as functions of Galactic longitude that are continuous across the groups.}

\subsection{Stellar Populations}

Motivated by our initial spectroscopic results for the M giant stars, in \cite{pricewhelan15} we selected RR Lyrae stars in TriAnd region from the Catalina Survey \citep{drake14}, that had appropriate apparent magnitudes to be possible members of the overdensity. 
We collected spectra for our sample in order to confirm which fell within the velocity sequence defined by the M giant stars with the aims of finding the distances to the TriAnd I and TriAnd II sequences more precisely and better constraining the models.

\begin{figure}[t]
\centering
%\includegraphics[width=3 cm]{logo-mdpi}
\caption{\label{fig:apw}
v vs l for M giants vs RR Lyrae}
\end{figure}   

Figure \ref{fig:apw} \citep[reproduced from][]{pricewhelan15} shows the results of our survey: unlike the M giants (gray points) the RR Lyrae (black points) showed no clear, tight velocity sequence. 
By modeling both the RR Lyrae and M giants velocity data as mixtures of two gaussians representing a cold foreground sequence and a possible background halo population of larger dispersion, we were able to show that the number ratio of RR Lyrae to M giants ($f_{\rm RR:M}$), within the overdensity was less than ??? with 95\% confidence.

In order to aid with the interpretation of this discovery, Figure \ref{fig:isochrones} plots isochrones for two 10 Gyear-old populations of low and intermediate metallicity, including the color range for RR Lyrae and M spectral class. 
It emphasizes our understanding that these two types of stars are tracers of populations with quite distinct metallicities and illustrates why, while the initial aim of our survey to find more accurate distances to TriAnd I and II had failed, our results could be even more intriguing.
The metallicity distributions in nearly all existing satellite galaxies orbiting the Milky Way \citep[e.g.][]{kirby11} are such that they contain no M giant stars (i.e. $f_{\rm RR:M} = \infty$).
The very largest satellites (the Large Magellanic Cloud and the Sagittarius dwarf galaxy), with the most metal rich populations, have $f_{\rm RR:M}\sim 0.5$ \citep{pricewhelan15}.
In contrast, the Galactic disk has very few RR Lyrae and  $f_{\rm RR:M} \sim 0$, most consistent with our findings \citep{amrose01}.

Our work on the TriAnd region motivated us to go on to look at possible associations of RR Lyrae with the M giant sequences we had found for Mon/GASS and A13  \cite{sheffield17}. 
We found $f_{\rm RR:M}$ values similar to those observed for TriAnd , and consistent with membership of the Galactic disk.

%{\it Summary: the stellar populations in these groups look like each other; they are more consistent with those in the Galactic disk rather than those observed in the stellar halo or Galactic  satellites.}

\subsection{Abundance Patterns}

\cite{chou2010b} took high-resolution spectra of 21 M giants in the GASS overdensity. They found that for some stars the abundances of titanium and yttrium deviate from that of the Milky Way disk stars at the same metallicity.  For Ti, the measured abundances are lower by up to 0.4 dex compared to the mean trends known for main-sequence stars of the Galactic disk (e.g Reddy et al. 2003, Bensby et al. 2014). The mean offset for [Y/Fe] is about 0.2 dex at [Fe/H] $\approx -0.5$. Comparing with their own results in Sgr dwarf spheroidal galaxy \citep{chou2010a} concluded that the abundances are consistent with the GASS stars having an external origin. It should be noted, however, that their results do not agree with the abundances of Ti and, in particular, La, measured in the Canis Major overdensity by \cite{sbordone2005}. Their analysis was based on spectra covering a limited spectral region in the near-IR, including 11 neutral iron and 2 neutral titanium lines. Also \cite{chou2010b} caution that the effects of non-local thermodynamic equilibrium (non-LTE) may be significant, since their measurement of metallicity and Ti abundances relies on the LTE analysis of lines of neutral species,

We have recently obtained high-resolution and signal-to-noise spectra of 15 stars in TriAnd1 and A13 overdensities. 14 stars were observed with the HIRES-S spectrograph at the Keck-1 telescope \cite{vogt1994} and 1 star was observed using the the UVES spectrograph at the VLT (Program ID: 097.B-0770A). The spectral resolving power R of the HIRES spectra is 36\,000 and the UVES data have R $\sim$ 47\,000. All Keck spectra cover the full optical region, from $4800$ to $8770$ $\Angstrom$, although the exact wavelength coverage varies as several slightly different instrument configurations were tried in an effort to get all the key lines into a single exposure. The signal-to-noise (SNR)/\Angstrom of the HIRES spectra near 5200 \Angstrom in the continuum at the center of the echelle order exceeds 200. For the UVES spectrum, the SNR of 50 near 5500 \Angstrom was achieved. The fundamental parameters and chemical abundances of stars were determined using 2MASS and APASS photometry and the high-resolution spectra. All 15 stars are M-type giants with effective temperatures Teff ~ 3800 K and surface gravities log(g) ~ 1 dex. The $T_{\rm eff}$ estimates were derived using the Infrared Flux Method (Casagrande et al. 2010, 2014). Chemical abundances were measured for 6 chemical elements, including O, Na, Mg, Ca, Fe, Eu using LTE and the standard MARCS stellar model atmospheres (Gustafsson et al. 2008). We have also performed detailed analysis of the effects of non-local thermodynamic equilibrium (non-LTE) for the elements, where detailed calculations are available \cite{bergemann2011, bergemann2012, bergemann2016}, however, the NLTE corrections for the chosen lines are minor and do not change our conclusions. In what follows, we use our LTE results, because all studies of chemical abundances in dSph systems and most studies in the Milky Way to date have employed LTE with 1D hydrostatic models

We find that stars in the TriAnd1 and in A13 overdensities have extemely similar chemical abundances, with the abundance dispersion $\leq 0.06$ dex for most chemical elements. The TriAnd1 and A13 stars also have a very narrow metallicity spread,  [Fe/H] $= -0.92 \pm 0.06$ dex. The abundances of all measured alpha-elements\footnote{The only exception is [Ca/Fe], which is solar with the dispersion of 0.06 dex; however, the NLTE abundance correction for the \ion{Ca}{i} lines for M giants is 0.1 dex that will bring the results in agreement with the MW disk studies that are based on FGK dwarfs, for which the NLTE correction is much smaller \cite{Merle2011}}. are uniformly enhanced at the level of 0.43 $\sim$ 0.06 dex for [O/Fe], 0.41 $\sim$ 0.02$ dex for [Mg/Fe], $0.35 \pm 0.06$ dex for [Na/Fe], and 0.15 $\sim$ 0.03 dex for [Eu/Fe] (Figure 1).However, the At this level, the TriAnd1 and A13 stars are consistent with the abundances of the in-situ formed Milky Way thick disk stars \cite{fuhrmann2002,bergemann2014,bensby2014}. The abundance ratios are too high for the chemical abundance patterns observed in the stars of the Galactic satellites (Bonifacio et al. 2010, Shetrone et al. 2001, 2003, Tolstoy et al. 2009, de Boer et al. 2014), which are known to have low, typically solar ( [O/Fe], [Mg/Fe]) or even sub-solar ([Na/Fe]), ratios at [Fe/H] $\sim -1$. For example, Fornax, a dSph galaxy most close to TriAnd1 and A13 in metallicity, has [Na/Fe] $\sim -0.6$ dex and [Mg/Fe] $\sim 0$ at [Fe/H] $=-1$ (Letarte et al. 2010, Kirby 2010, Lemasle et al. 2014) \footnote{Note that to avoid systematic differences between abundance measurements obtained from spectra taken with different instruments, we have chosen here to compare with the data taken with the same instrument; here UVES and FLAMES at the VLT, ESO, or with HIRES at Keck.}. Thus it is unlikely that the TriAnd1 and A13 stars originate from a disrupted dSph galaxy.

Our results for TriAnd and A13 do not confirm the conclusions by \cite{chou2010b}. Although there could be several astrophysical reasons for the differences, this could be caused by the differences in the observed datasets and spectroscopic modelling techniques. Our observed data cover the full optical range from the near-UV to the near-IR, whereas \cite{chou2010b} have analysed only the small wavelength region in the near-IR, limited to 150 \Angstrom from 7440 to 7590 \Angstrom. Because of this limitation, they could include only 11 \ion{Fe}{I} and 2 \ion{Ti}{I} lines in the determination of gravities, metallicities, and abundances. Our analysis, through a much wider wavelength coverage and high SNR attained for the observed spectra, allowed us to include 70 lines of \ion{Fe}{I} and \ion{Fe}{II} \cite{bergemann2012}, as well as 10 lines of \ion{Ti}{2}. \cite{bergemann2011} showed that \ion{Ti}{1} should be not be used in abundance studies because it is very sensitive to NLTE effects. We have also performed test computations, using a reduced linelist, to explore the sensitivity of abundance diagnostic to the line selection and wavelength regimes. We have found that using the linelist from \cite{chou2010a}, the metallicities are over-estimated by $\sim 0.25$ dex and [Ti/Fe] are under-estimated by $0.4$ dex, compared to the results using the complete linelist. This suggests that the choice of the linelist and diagnostic spectral band (full optical or near-IR) could be a possible reason for the difference between us and \cite{chou2010}.
%{\it Summary: the abundance patterns  of the low-latitude structures are similar to those of the thick disk of our Galaxy.}

\subsection{Local Spectroscopic Surveys}

Large scale spectroscopic surveys have allowed a detailed re-examination of the distribution of stellar velocities.
Asymmetries between the Northern and Southern Galactic hemispheres in the vicinity of the Sun have been seen in the density and velocity distributions using data from SDSS \citep{widrow12,yanny13} and RAdial Velocity Experiment \citep[RAVE, see][]{??,williams13}. 
Looking $\sim$2kpc out towards the Galactic anticenter, the Large Sky Area Multi-Object Fiber Spectroscopic Telescope \citep[LAMOST,][]{cui12,demg12,zhao12} finds similar asymmetries in radial and vertical velocities \citep{carlin13}.
The scale and sense of these asymmetries indicate moderate systematic motions (of order a few km/s)  of stars within the disk perpendicular to the plane,   suggesting both vertical movement of midplane, and compression and rarefaction of the vertical scale \citep[referred to as ``bending'' and ``breathing'' modes respectively --- see,][]{widrow14}.
  
%{\it Summary: small-scale, systematic vertical motions of and within the disk have been detected in the Solar Neighborhood.}

\subsection{Photometric Surveys}

Co-incidental with the above spectroscopic and stellar populations work, \cite{xu15} and \cite{jurie17} explored the spatial distribution of stars in the anticenter region using photometry in the SDSS and PanSTARRS \citep{kaiser10} respectively. 
Both employed the novel technique of subtracting color-magnitude diagrams (CMD's) derived from fields in their photometric data which were symmetrically placed at equal and opposite Galactic latitudes and at the same Galactic longitudes.
These differenced CMD's showed sequences in overdensities oscillating between the northern and southern hemispheres as the distance from the Sun was increased towards the anticenter of our Galaxy.
The vast numbers of stars in these surveys allowed more clear identification of the global structures as clearly distinct, separate features.

%{\it Summary: the overdensities around the outer Galaxy, oscillating between the northern and southern hemispheres have been traced to smaller-scale oscillations all the way to the Solar Neighborhood.}



\section{The Nature of Structures Around the Outer Disk --- Summary of Theoretical Interpretations}
\label{sec:theory}

The observations summarized in Section \ref{sec:obs} indicate that:
\begin{itemize}
\item the low latitude structures --- Mon//GASS, TriAnd I/II and A13  --- each have low velocity dispersions supporting the genuine association of the stars within them; 
\item Mon//GASS, TriAnd I/II and A13 also share a single, continuous sequence in average velocity as a function of $l$ suggestive of associations across these groups;
\item the stellar populations (as indicated by $f_{\rm RR:M}$) in these groups look like each other; 
\item the stellar populations are more consistent with those in the Galactic disk rather than those observed in the stellar halo or Galactic  satellites;
\item the abundance patterns  of the low-latitude structures are similar to those of the thick disk of our Galaxy;
\item small-scale, systematic vertical motions of and within the disk have been detected in the Solar Neighborhood which suggest the local disk is moving up and down;
\item the low-latitude structures around the outer Galaxy connect to overdensities oscillating between the northern and southern hemispheres on smaller scales and traced all the way back to the Solar Neighborhood.
\end{itemize}

\begin{figure}[t]
\centering
%\includegraphics[width=3 cm]{logo-mdpi}
\caption{\label{fig:cartoon}
Cartoon of oscillations in space and velocity}
\end{figure}   

Figure \ref{fig:cartoon} is a cartoon which summarizes the observational work and its implications, by showing the approximate locations and amplitudes of the spatial and velocity structures that have been identified. (Note that the figure is misleading as the structures are {\it not} consistent morphologically with concentric rings that are axisymmetric about the Galactic center.)
Taken together, we conclude: (i) there is compelling evidence that Mon/GASS, TriAnd I/II and A13 represent populations of stars formed in the disk that now exist at extreme $(R,Z)$; 
(ii) that these structures are associated with each other as part of a global system of vertical disk oscillations that can be traced all the way to the velocity asymmetries seen in the Solar Neighborhood; and 
(iii) that the stellar populations in these structures are inconsistent with a picture where they formed from an accreted satellite.

The natural interpretation of these collected observations is that the oscillations represent the response of the disk to an external perturbation, for example as the impact of a satellite galaxy is transmitted and amplified by its wake in the dark matter halo \citep[as described for the LMC in][]{weinberg06}
Prior work has already pointed to this as a possible explanation for the existence of Mon/GASS \citep{katzantzidis08,younger08}, with the Sagittarius dwarf galaxy being pointed to as a plausible culprit for the perturbation \citep{purcell11}.
 \cite{widrow14} demonstrated how perturbations from a satellite on an orbit perpendicular to the disk could lead to bending (at low relative impact velocity) and breathing (at higher relative velocity) modes that would be observed in the Solar Neighborhood as asymmetries in the local velocity distribution and simulations have also shown that Sgr could be responsible for this local velocity structure  \cite{gomez13}.
Such interactions and corresponding disk features have been found to naturally occur in cosmological simulations \citep{gomez16}.

\begin{figure}[t]
\centering
%\includegraphics[width=3 cm]{logo-mdpi}
\caption{\label{fig:chervin}
Pretty plots form Chervin}
\end{figure}   

Figure \ref{fig:chervin} illustrates simulations from our own recent work.
We extended the prior theoretical backdrop to examine whether the extreme locations of TriAnd I/II could fit within the same picture using simulations of a disk disturbed (separately) by satellites on orbits that mimic those expected for the Large Magellanic Cloud and Sagittarius \citep{laporte17a}. 
With different masses and orbits (interaction timings, and frequency and distances of pericentric passages) these satellite necessarily induced distinct but overlapping signatures.
In \cite{laporte17b} we found that a model that was capable of reproducing the scales of the observed disturbances (radial wavelength and  amplitude in space, as well as magnitude of offsets in velocity locally) needed:
the  interaction of Sgr with the disk of the Milky Way to be followed for several passages longer than prior work; 
Sgr to have of sufficient initial mass and density to impact the disk in the last Gigayear with a remaining mass of ???;
and the disk to be realized with stars existing as far out as ???kpc form the Galactic center in order to populate the regions corresponding to TriAnd I/II.
The interaction with the LMC modified the overall morphology of the structures induced, but was not sufficient alone to explain their properties.

\section{Discussion --- Observational Prospects}

\subsection{The Milky Way}

%It is interesting to place the current work in the context of ongoing and near-future studies of our Galaxy.
While the connections that have already been made between the different structures and of these structures with the disk population are convincing, there are several directions for further observations which would greatly facilitate an informative comparison to theoretical work, allowing more detailed interpretations and tighter constraints on matter distributions and histories.

The most obvious first direction would be to increase the dimensions of motion and accuracy of distance estimates to known features.
For example, \cite{carlin10} report proper motions for the ``Anti-Center Stream'' (ACS), which may or may not be part of the larger GASS/Mon structure \citep[it is at the right distance but is morphologically distinct, see][]{li12}.
These proper motions indicate that stars in the ACS are not actually moving parallel to the stream, which is inconsistent with expectations for the behavior of debris from a destroyed satellite. 
If similar measurements of proper motions of stars in all the low-latitude structures showed significant motion perpendicular to the Galactic disk, this would provide the final evidence of a disk origin and connection to local oscillations, as well as important constraints on dynamical models (see Section \ref{sec:conc}).
In near-future data-releases, ESA's {\it Gaia} satellite \cite{} is poised to provide just such proper motion data --- with proper motion accuracies of ??? $\mu$as/year or ?? km/s for our (closest) ??? mag M giant stars in Mon/GASS and ??? $\mu$as/year or ?? km/s for our (farthest) ??? mag M giant stars in TriAnd.

The second direction is to continue the maps first made by \cite{xi15} and \cite{lurie17} to global scales. Again, {\it Gaia} will be able to tackle this with distances, proper motions and radial velocities, although its reach towards low latitudes in the inner Galaxy may be limited by extinction.
DISCO/APOGEE could overcome this limitation with an infrared spectroscopic survey of the disk capable of reaching entirely across the Galaxy.

\subsection{Other Galaxies}

The great advantage of star-count studies is the ability to reach extremely low surface brightness levels.
\cite{majewski04} estimate a surface brightness for TriAnd with $\Sigma <$32 mag/arcsec$^2$....
%For example, TriAnd I contains ?? M-giant stars spread over an area on the sky of roughly ????, so ? giants/deg$^2$.
%Adopting a 10 Gyr-old isochrone for this population \citep[from the Padova group][]{}, each M-giant has an associated total stellar luminosity of ??? in the ?? band. Hence, the equivalanet urfc ebright ness would be .....

The growing samples of galaxies within and beyond the Local Group being mapped to extremely low surface brightness levels are intriguing.
Nearby, these can be reached, like the Milky Way, through star counts studies, most spectacularly for the case of our nearest neighbor, the Andromeda Galaxy, where giant star counts have revealed a significantly extended and richly featured outer stellar disk \citep{ferguson02,ibata05}.
Analogous studies have been carried out for galaxies up to distances of several Mpc \citep[e.g. the GHOSTS survey +???][]{}, although the focus of these studies has typically been on detecting the stellar halo of these objects.
Surveys using integrated light are also emerging \citep[e.g. Dragonfly and ATLAS3D][]{}, although these typically are unable to go quite as deep (limits of $\mu <$30 mag/arcsec$^2$).
Martinez-Delgado.....

Looking to the future, the power of star count studies - wide field and space! WFIRST
Large Synoptic Survey Telescope .....



\section{Conclusion --- What Might These Structures Tell Us About Galaxies?} 

{\it KVJ --- Reminder to self to add these somewhere} \\
arXiv:1706.01900  --- Title: Milky Way Tomography with K and M Dwarf Stars: the Vertical Structure of
 the Galactic Disk \\
 D'Onghia et al 2016 on sat and disk interaction \\
 Bovy et al 2015 - power spectrum of vel in disk \\
 Kazantzidis08 - rings etc \\
 schwarzkopf 01 \\
 Zarik=tsky 97 - lopsided gals and accreiton
 


The above sections  summarized observational evidence for large scale vertical oscillations of the Galactic plane present in the Solar Neighborhood and reaching out beyond the traditional edge of our stellar disk.
They also collated theoretical studies that suggest that these oscillations could be caused by, and contain the signatures of, ongoing interactions of the Milky Way with its satellite system.
Moreover, observational prospects are bright for extending this work both to globally map the Milky Way, and to look for analogous features around many other galaxies.

Now that we have a physical picture of the origin of such features, as well as prospects for mapping them further within the Milky Way and detecting analogous substructures around other galaxies, we can move on to discussing how useful they are as probes of dynamics and history.
It is important to have these uses in mind to motivate and frame  future work.
While the mere existence of these structures is interesting, they contain a tiny fraction of the stars in galaxies spread out over a large area --- these properties naturally make them difficult to map, either because their unique signatures can be lost in the foreground star counts (e.g. in the Milky Way) , or  because the required surface brightness limits for detection are prohibitively low (for integrated light).

Conversely, these features around the outskirts of galaxies may prove to be particularly powerful probes of interactions and histories, precisely because they contain so little mass --- they can be modeled rather simply as test particles responding to an external perturbation.

Below are just three examples of where these structures could promise new insights into some classic questions.
\begin{description}
\item{\it Disk heating mechanisms ---}
It has been understood for a long time that disks can evolve significantly due to mergers, major or minor, and hence that their current structures bear witness to their accretion history \citep{toth92,quinn93,walker96,velasquez99}. 
This understanding has fueled a significant literature on the importance of the heating of galactic disks in response to encounters with other dark matter halos (that may or may not contain their own galaxies) \citep{font01,ardi03,benson04,stewart08,hopkins08,villalobos08,purcell09,kazantzidis09,sachdeva16,moetazedian16}.
In general, these studies have concentrated on the overall effects of many encounters on global properties, such as the thickness and vertical velocity dispersion in disks.
Their results have traditionally been compared to these scales in samples of galaxies.
In contrast, the identification and mapping of vertical waves in our Milky Way associated with ongoing interactions gives us the opportunity to dissect an individual disk heating event in progress.
We can use this to check our understanding of the mechanism directly and in detail rather than assessing its importance through collective effects and longterm, phase-mixed signatures.
\item{\it Stellar halo formation processes ----}
The last decade has seen increasing interest in assessing how much of the content of stellar halos could be made from stars originally formed in the disks of the galaxies that they surround. Hydrodynamical simulations of galaxy formaion suggest that tens of percent of the inner halo might be formed this way
\citep{abadi06,zolotov09,zolotov10,font11,mccarthy12,tissera13,tissera14,pillepich15,cooper15}.
Preliminary arguments for the existence of this ``kicked-out-disk'' population were based on transitions in the density or orbital structures of stellar halos \citep[e.e.][]{carollo07}.
However, such transitions were also found to naturally occur in purely-accreted models of stellar halos \citep{deason13}.
More convincing observational evidence for disk stars in the halo is just beginning to emerge through studies with look for stars with halo-like kinematics, but disk-like abundances around M31 \citep{dorman13} and the Milky Way \citep{sheffiled12,hawkins15,bonaca17}.
Our own work adds new perspectives on this stellar halo formation process with the detection and modeling of disk stars that may be in transition from the disk to the halo.
\item{\it Galactoseismic probes of interactions and dark matter ---}
The response of a disk to an encounter will depend on its own properties, the properties of the dark matter halo in which it is embedded and the mass and orbit of the perturbing satellite.
This leads to the suggestion that, analogous to helioseismic investigations of the structure of our Sun, maps of a disk response --- such as those described in Section \ref{sec:obs} for our Milky Way --- might be similarly inverted to tell us about (e.g.) the structure of the dark matter halo
\citep{widrow11}.
Indeed, recent studies of the signatures of encounters in the very outskirts of extended HI disks have successfully used simulations combined with an analytic understanding to find how the observed characteristics of the disturbed gas can be simply related to properties of the perturbing object \citep{chakrabarti09,chakrabart11a,chakrabarti11b,chang11}.
\end{description}

Uplifitng/Concluding sentence.

%%%%%%%%%%%%%%%%%%%%%%%%%%%%%%%%%%%%%%%%%%
\vspace{6pt} 

%%%%%%%%%%%%%%%%%%%%%%%%%%%%%%%%%%%%%%%%%%
%% optional
%\supplementary{The following are available online at www.mdpi.com/link, Figure S1: title, Table S1: title, Video S1: title.}

%%%%%%%%%%%%%%%%%%%%%%%%%%%%%%%%%%%%%%%%%%
\acknowledgments{All sources of funding of the study should be disclosed. Please clearly indicate grants that you have received in support of your research work. Clearly state if you received funds for covering the costs to publish in open access.}

%%%%%%%%%%%%%%%%%%%%%%%%%%%%%%%%%%%%%%%%%%
\authorcontributions{For research articles with several authors, a short paragraph specifying their individual contributions must be provided. The following statements should be used ``X.X. and Y.Y. conceived and designed the experiments; X.X. performed the experiments; X.X. and Y.Y. analyzed the data; W.W. contributed reagents/materials/analysis tools; Y.Y. wrote the paper.'' Authorship must be limited to those who have contributed substantially to the work reported.}

%%%%%%%%%%%%%%%%%%%%%%%%%%%%%%%%%%%%%%%%%%
\conflictsofinterest{The authors declare no conflict of interest.} 

%%%%%%%%%%%%%%%%%%%%%%%%%%%%%%%%%%%%%%%%%%
%% optional
%\abbreviations{The following abbreviations are used in this manuscript:\\

%\noindent 
%\begin{tabular}{@{}ll}
%MDPI & Multidisciplinary Digital Publishing Institute\\
%DOAJ & Directory of open access journals\\
%TLA & Three letter acronym\\
%LD & linear dichroism
%\end{tabular}}

%%%%%%%%%%%%%%%%%%%%%%%%%%%%%%%%%%%%%%%%%%
%%%%%%%%%%%%%%%%%%%%%%%%%%%%%%%%%%%%%%%%%%
% Citations and References in Supplementary files are permitted provided that they also appear in the reference list here. 

%=====================================
% References, variant A: internal bibliography
%=====================================
\bibliography{refs_1.bib,refs_2.bib,refs_3.bib}

%\begin{thebibliography}{999}
% Reference 1
%\bibitem[Author1(year)]{ref-journal}
%Author1, T. The title of the cited article. {\em Journal Abbreviation} {\bf 2008}, {\em 10}, 142-149.
% Reference 2
%\%bibitem[Author2(year)]{ref-book}
%Author2, L. The title of the cited contribution. In {\em The Book Title}; Editor1, F., Editor2, A., Eds.; %Publishing House: City, Country, 2007; pp. 32-58.
%\end{thebibliography}


%%%%%%%%%%%%%%%%%%%%%%%%%%%%%%%%%%%%%%%%%%
\end{document}

