\documentclass[galaxies,article,submit,moreauthors,pdftex,10pt,a4paper]{mdpi}

% TODO:
% - Mention Hercules-Aquila somewhere as a genuine "shell" candidate

\usepackage{amsmath}
\usepackage{amssymb}
\usepackage{aas_macros}

% -------------------
% Some useful macros:
% -------------------

% Formatting:
\newcommand{\project}[1]{\textsl{#1}}
\newcommand{\survey}[1]{\textsl{#1}}
\newcommand{\acronym}[1]{{\small{#1}}}
\newcommand{\apogee}{\survey{\acronym{APOGEE}}}
\newcommand{\sdssiii}{\survey{\acronym{SDSS-III}}}
\newcommand{\documentname}{\emph{Article}}

% Math:
\newcommand{\kpc}{\mathrm{kpc}}
\newcommand{\msun}{\mathrm{M}_\odot}
\newcommand{\kms}{\mathrm{km}~\mathrm{s}^{-1}}
\newcommand{\frrmg}{\ensuremath{f_{\rm RR:MG}}}

\newcommand{\feh}{\ensuremath{[{\rm Fe}/{\rm H}]}}
\newcommand{\ion}[2]{#1\,\textsc{#2}}

%--------------------
% Class Options:
%--------------------
% journal
%----------
% Choose between the following MDPI journals:
% actuators, admsci, aerospace, agriculture, agronomy, algorithms, animals, antibiotics, antibodies, antioxidants, applsci, arts, atmosphere, atoms, axioms, batteries, bdcc, behavsci, beverages, bioengineering, biology, biomedicines, biomimetics, biomolecules, biosensors, brainsci, buildings, carbon, cancers, catalysts, cells, challenges, chemengineering, chemosensors, children, chromatography, climate, coatings, computation, computers, condensedmatter, cosmetics, cryptography, crystals, data, dentistry, designs, diagnostics, diseases, diversity, econometrics, economies, education, electronics, energies, entropy, environments, epigenomes, fermentation, fibers, fishes, fluids, foods, forests, fractalfract, futureinternet, galaxies, games, gastrointestdisord, gels, genealogy, genes, geosciences, geriatrics, healthcare, horticulturae, humanities, hydrology, informatics, information, infrastructures, inorganics, insects, instruments, ijerph, ijfs, ijms, ijgi, ijtpp, inventions, jcdd, jcm, jcs, jdb, jfb, jfmk, jimaging, jof, jintelligence, jlpea, jmmp, jmse, jpm, jrfm, jsan, land, languages, laws, life, literature, logistics, lubricants, machines, magnetochemistry, marinedrugs, materials, mathematics, mca, mti, medsci, medicines, membranes, metabolites, metals, microarrays, micromachines, microorganisms, minerals, molbank, molecules, mps, nanomaterials, ncrna, neonatalscreening, nitrogen, nutrients, ohbm, particles, pathogens, pharmaceuticals, pharmaceutics, pharmacy, philosophies, photonics, plants, polymers, proceedings, processes, proteomes, publications, quaternary, qubs, recycling, religions, remotesensing, resources, risks, robotics, safety, scipharm, sensors, separations, sexes, sinusitis, socsci, societies, soilprocesses, soils, sports, standards, sustainability, symmetry, systems, technologies, toxics, toxins, tropicalmed, universe, urbansci, vaccines, vetsci, viruses, vision, water
%---------
% article
%---------
% The default type of manuscript is article, but can be replaced by:
% addendum, article, benchmark, book, bookreview, briefreport, casereport, changes, comment, commentary, communication, conceptpaper, correction, conferenceproceedings, conferencereport, expressionofconcern, meetingreport, creative, datadescriptor, discussion, editorial, essay, erratum, hypothesis, interestingimage, letter, newbookreceived, opinion, obituary, projectreport, reply, reprint, retraction, review, perspective, preprints, protocol, shortnote, supfile, technicalnote, viewpoint
% supfile = supplementary materials
%----------
% submit
%----------
% The class option "submit" will be changed to "accept" by the Editorial Office when the paper is accepted. This will only make changes to the frontpage (e.g. the logo of the journal will get visible), the headings, and the copyright information. Also, line numbering will be removed. Journal info and pagination for accepted papers will also be assigned by the Editorial Office.
%------------------
% moreauthors
%------------------
% If there is only one author the class option oneauthor should be used. Otherwise use the class option moreauthors.
%---------
% pdftex
%---------
% The option pdftex is for use with pdfLaTeX. If eps figure are used, remove the option pdftex and use LaTeX and dvi2pdf.

%=================================================================
\firstpage{1}
\makeatletter
\setcounter{page}{\@firstpage}
\makeatother
\articlenumber{x}
\doinum{10.3390/------}
\pubvolume{xx}
\pubyear{2017}
\copyrightyear{2017}
\externaleditor{Academic Editor: name}
\history{Received: date; Accepted: date; Published: date}

%------------------------------------------------------------------
% The following line should be uncommented if the LaTeX file is uploaded to arXiv.org
%\pdfoutput=1

%=================================================================
% Add packages and commands here. The following packages are loaded in our class file: fontenc, calc, indentfirst, fancyhdr, graphicx, lastpage, ifthen, lineno, float, amsmath, setspace, enumitem, mathpazo, booktabs, titlesec, etoolbox, amsthm, hyphenat, natbib, hyperref, footmisc, geometry, caption, url, mdframed, tabto, soul, multirow, microtype

%=================================================================
%% Please use the following mathematics environments: Theorem, Lemma, Corollary, Proposition, Characterization, Property, Problem, Example, ExamplesandDefinitions, Hypothesis, Remark, Definition
%% For proofs, please use the proof environment (the amsthm package is loaded by the MDPI class).

%=================================================================
% Full title of the paper (Capitalized)
\Title{Disk Heating, Galactoseismology, and the Formation of Stellar Halos}

% If this is an expanded version of a conference paper, please cite it here: enter the full citation of your conference paper, and add $^\dagger$ in the end of the title of this article.
%\conference{Title}

% Authors, for the paper (add full first names)
\Author{Kathryn V. Johnston $^{1,\dagger}$, Adrian M. Price-Whelan$^{2,\dagger}$, Maria Bergemann$^{3}$, Chervin Laporte$^{1}$, Ting S. Li$^{4}$, Allyson A. Sheffield$^{5}$, Branimir Sesar$^{3}$, and Sanjib Sharma $^{6}$}
% Rachel Beaton? Steve Majewski? Sanjib Sharma?

% Authors, for metadata in PDF
\AuthorNames{Firstname Lastname, Firstname Lastname and Firstname Lastname}

% Affiliations / Addresses (Add [1] after \address if there is only one affiliation.)
\address{
$^{1}$ \quad Department of Astronomy, Columbia University, 550 W 120th st., Mail Code 5246, New York, NY, 10027 USA \\
$^{2}$ \quad Department of Astrophysical Sciences, Princeton University, 4 Ivy Lane, Princeton, NJ 08544, USA \\
$^{3}$ \quad Max Planck Institute for Astronomy, Heidelberg\\
$^{4}$ \quad Fermi National Accelerator Laboratory, P. O. Box 500, Batavia, IL 60510, USA\\
$^{5}$ \quad Department of Natural Sciences, LaGuardia Community College, City University of New York, 31-10 Thomson Ave., Long Island City, NY 11101, USA \\
$^{6}$ \quad Sydney Institute for Astronomy, School of Physics, University of Sydney \\
}

% Contact information of the corresponding author
\corres{Correspondence: kvj@astro.columbia.edu; Tel.: +1-212-854-3884}

% Current address and/or shared authorship
%\firstnote{Current address: Affiliation 3}
\firstnote{These authors contributed equally to this work.}
% The commands \thirdnote{} till \eighthnote{} are available for further notes

% Simple summary
%\simplesumm{}

% Abstract (Do not use inserted blank lines, i.e. \\)
\abstract{
%A single paragraph of about 200 words maximum. 1) Background: Place the question addressed in a broad context and highlight the purpose of the study; 2) Methods: Describe briefly the main methods or treatments applied; 3) Results: Summarize the article's main findings; and 4) Conclusion: Indicate the main conclusions or interpretations. The abstract should be an objective representation of the article, it must not contain results which are not presented and substantiated in the main text and should not exaggerate the main conclusions.
Deep photometric surveys of the Milky Way have revealed diffuse structures
encircling our Galaxy far beyond the ``classical'' end of the stellar disk.
This \documentname\ reviews results from our own and other observational programs, which together suggest that, despite their extreme positions, the stars in these structures were formed in our Galactic disk.
Mounting evidence from recent observations and simulations implies kinematic connections between several of these distinct structures.
This suggests the existence of collective disk oscillations that can plausibly be traced all the way to asymmetries seen in the stellar velocity distribution around the Sun.
There are multiple interesting implications of these findings:
they promise new perspectives on the process of disk heating;
they provide direct evidence for a stellar halo formation mechanism in addition to the accretion and disruption of satellite galaxies; and,
they motivate searches of current and near-future surveys to trace these oscillations across the Galaxy.
Such maps could be used as dynamical diagnostics in the emerging field of ``Galactoseismology,'' which promises to model the history of interactions between the Milky Way and its entourage of satellites, as well examine the density of our dark matter halo.
As sensitivity to very low surface brightness features around external galaxies increases, many more examples of such disk oscillations will likely be identified.
Statistical samples of such features not only encode detailed information about
interaction rates and mergers, but also about long sought-after dark matter
halo densities and shapes.
Models for the Milky Way's own Galactoseismic history will therefore serve as a
critical foundation for studying the weak interactions of galaxies across the
universe.
}

% Keywords
\keyword{galaxy formation; galactic disks; stellar halos; Milky Way.}

% The fields PACS, MSC, and JEL may be left empty or commented out if not applicable
%\PACS{J0101}
%\MSC{}
%\JEL{}

%%%%%%%%%%%%%%%%%%%%%%%%%%%%%%%%%%%%%%%%%%
% Only for journal Applied Sciences:
%\featuredapplication{Authors are encouraged to provide a concise description of the specific application or a potential application of the work. This section is not mandatory.}
%%%%%%%%%%%%%%%%%%%%%%%%%%%%%%%%%%%%%%%%%%


%%%%%%%%%%%%%%%%%%%%%%%%%%%%%%%%%%%%%%%%%%
% Only for the journal Data:
%\dataset{DOI number or link to the deposited data set in cases where the data set is published or set to be published separately. If the data set is submitted and will be published as a supplement to this paper in the journal Data, this field will be filled by the editors of the journal. In this case, please make sure to submit the data set as a supplement when entering your manuscript into our manuscript editorial system.}

%\datasetlicense{license under which the data set is made available (CC0, CC-BY, CC-BY-SA, CC-BY-NC, etc.)}

%%%%%%%%%%%%%%%%%%%%%%%%%%%%%%%%%%%%%%%%%%
% For Conference Proceedings Papers:
%\conferencetitle{Add the conference title here}

%\setcounter{secnumdepth}{4}
%%%%%%%%%%%%%%%%%%%%%%%%%%%%%%%%%%%%%%%%%%

\begin{document}

%%%%%%%%%%%%%%%%%%%%%%%%%%%%%%%%%%%%%%%%%%

%%%%%%%%%%%%%%%%%%%%%%%%%%%%%%%%%%%%%%%%%%
%\setcounter{section}{-1} %% Remove this when starting to work on the template.

\section{Introduction}

Our perspective on the Milky Way presents both unique challenges and unique opportunities towards our quest to understand galaxies more generally.
From our position inside, it is the one galaxy in the Universe that we cannot take an image of but rather need to survey the entire sky in order to map its structure.
On the other hand, it is one of the few galaxies that we can presently study by individual stars and is the only one for which we can hope to make volume-complete maps of full-space positions and velocities for significant numbers of unevolved stars.
Present and recent sky surveys have already considerably advanced this effort, and near-future surveys will deliver massive datasets that will enable detailed studies of stellar structures throughout the Galactic volume.
%move beyond co-ordinates restricted to random projections in a limited number of directions to

% We are in the middle of a renaissance in Milky Way studies, fueled by stellar data sets of sufficient scope in terms of sky coverage and numbers to exploit our perspective as a strength rather than a liability
% %and take full advantage of our proximity
% to create vast catalogues of stellar data.

Emerging in the 1990's, the catalogues that inspired this recent acceleration not only mapped global structures in our Galaxy, but also revealed the ubiquity of substructure within it.
These revelations added an unforeseen richness to interpretations and encouraged the development of new dynamical tools for studying ongoing interactions and formation histories.
As a few examples:
\begin{itemize}
    \item Astrometric data from the \survey{Hipparcos} mission \cite{esa97} led
        to the discovery of moving groups of stars in the velocity distribution
        of solar-neighborhood stars \cite{dehnen98}. Some of these likely
        correspond to destroyed star clusters (as expected), while others
        (unexpectedly) have been interpreted as signatures of resonances with
        the Galactic bar \cite{dehnen00}.
    \item Precise, large-area photometry from the Sloan Digital Sky Survey
        \cite[hereafter, SDSS ---][]{york00,stoughton02,abazajian03} led to the
        discovery of many ``streams'' of Main Sequence Turnoff (MSTO) stars in
        the Galactic stellar halo. These are understood to be the remnants of
        long-dead satellite galaxies and dissolved globular clusters
        \cite{newberg02,belokurov06} and serve as a stunning confirmation that
        our Galaxy had indeed formed hierarchically
        \cite[e.g.,][]{bullock01,bullock05}.
    \item All-sky, infrared photometry from the Two Micron All Sky Survey
        \cite[hereafter, 2MASS ---][]{nikolaev00} enabled tracing M giant stars
        associated with tidal debris from the Sagittarius dwarf galaxy around
        the entire Sky \cite{majewski03}, offering a new perspective on the
        history of its disruption \cite{law10}.
\end{itemize}
From these and many other studies that use large survey catalogues, it is now clear that the Milky Way is full of kinematic substructure, from the nearby regions of the Galactic disk to the distant stellar halo.

\begin{figure}[!ht]
\label{fig:ting}
\centering
\includegraphics[width=6 in]{figures/xy_Rz.pdf}
\caption{\label{fig:ting}
Summary of the spatial distribution of M giants in each of the three
low-latitude structures.
Note that at lower Galactic latitudes, the lack of candidate M giant members is
due to selection effects and crowding near the midplane; we expect the
structures to continue towards even lower latitudes, but blend with ordinary
disk stars.
Markers represent individual stars identified as likely members of each of the
three structures discussed in this work (see figure legend).
Distance estimates come from photometric alone and have expected absolute
uncertainties around $\approx$20\% for TriAnd and A13 \cite{sheffield14,li17}
and $\approx$25\% for GASS.
Grey curves in left panel show Galactocentric circles with cylindrical radii,
$R$, indicated on the figure.
The position of the sun is marked with the solar symbol $\odot$.
}
\end{figure}

The focus of this \documentname\ is on three such substructures just beyond the
historical ``end'' of the Galactic disk within the inner stellar halo.
Figure~\ref{fig:ting} \cite[reproduced with data from previous work, ][]{li17},
shows the spatial distribution of M giant stars associated with these three
substructures: the so-called Monoceros Ring (Mon; also known as the Giant
Anticenter Stellar Structure, GASS), the Triangulum-Andromeda clouds (TriAnd),
and A13.
Each of these substructures were originally identified as over-densities in
stellar number counts relative to the expected global structure of the disk or
inner stellar halo.\footnote{Here we show just M giant stars that have been
previously identified as candidate members of the structures, however, some of
the structures have also been detected in MSTO stars.}
The same region of the sky has been shown to be richly structured on even
smaller scales \cite{slater14,martin14,deason14}, but here we consider only the
larger structures.
Unlike most stellar streams, Mon/GASS, TriAnd, and A13 are present at a range
of low to moderate Galactic latitudes and span large areas of the sky --- we
will hereafter refer to them collectively as the ``low-latitude structures''.
The basic properties of the low-latitude structures are summarized below:
\begin{description}
\item{\it Mon/GASS} is an arc-like or partial-ring feature of stars beyond the previously-expected edge of the Galactic disk, $\approx 5~\kpc$ beyond the Sun in cylindrical radius \cite{robin92}.
Stars attributed to this feature span a large area of the sky and large range
in distance: Galactic longitudes $\approx 120^\circ \lesssim l \lesssim
240^\circ$, latitudes $-30^\circ \lesssim b \lesssim +40^\circ$, and
Heliocentric distances $5\lesssim d_\odot \lesssim 10~\kpc$
\cite{Morganson:2016}.
Radial velocity measurements of M giant stars associated with the structure
follow a clear trend in mean velocity with Galactic longitude and have a
velocity dispersion much smaller than the stellar halo \cite{crane03}.
\item{\it TriAnd} was first discovered as a diffuse over-density of M giant
stars covering the area $\approx 100^\circ \lesssim l \lesssim 150^\circ$,
$\approx 20^\circ \lesssim b \lesssim 40^\circ$, overlapping the Mon/GASS
structure on the sky but at larger Heliocentric distances of $\approx
15$--$25~\kpc$ \cite{rochapinto04}.
The M giants again exhibit a coherent radial-velocity sequence with a
dispersion much smaller than the halo \cite{rochapinto04}.
Deep photometry in the region revealed MSTO stars associated with the structure
and proposed the existence of a second main sequence (``TriAnd 2'') at larger
distance, $\approx 30$--$35~\kpc$ \cite{martin07,martin14}.
\item{\it A13} is another tenuous association of M giants in the north Galactic
hemisphere in the area $\approx 125^\circ \lesssim l \lesssim 210^\circ$,
$\approx 20^\circ \lesssim b \lesssim 40^\circ$ at approximate distances of
$\approx 12$--$20~\kpc$.
It was initially discovered by applying a group finding algorithm
\cite{sharma09} to all M giants in the 2MASS photometric catalogue
\cite{sharma10}.
Again, radial velocities of M giants in this structure have a small velocity
dispersion around a roughly linear trend with Galactic longitude \cite{li17}
\end{description}

Several distinct scenarios have been used to explain the formation and
existence of each of these structures.
Mon/GASS has been attributed to the accretion of a satellite on a retrograde
orbit \cite{penarrubia05}, a natural extension of the Galactic warp
\cite{momany04,momany06}, or disturbances to the Galactic disk
\cite{kazantzidis08,younger08,purcell11,xu15,gomez16}.
The extreme position of TriAnd at $(R,Z) \approx (15, -5)$ to $(25,-12)~\kpc$
across a large range in Galactocentric azimuth, $\phi$, seemed to exclude the
possibility of a disturbed Galactic disk as a possible origin and has also been
modeled as debris from a satellite on a retrograde orbit \cite{sheffield14}.
Initial abundance studies measured $[\alpha/{\rm Fe}]$ values for M giants in
both Mon/GASS and TriAnd and found them to be consistent with those seen in
Milky Way satellite galaxies, thus unlike the disk \cite{chou2010b,chou11} ---
see Section~\ref{sec:abundances} for a more complete discussion.

Recent evidence points towards a more convincing, coherent picture for the
nature of the three low-latitude structures: the stars in these structures
likely have a common origin in the Galactic disk and have been ``kicked out''
to their present-day positions.
This \documentname\ reviews contributions that our own group has made to formulating this picture, which include spectroscopic surveys of the low-latitude structures to study metallicities and kinematic properties \cite{sheffield14,li17}, stellar populations \cite{pricewhelan15}, and detailed abundance patterns (Bergemann et al., in prep.), as well as numerical simulations \cite{sheffield14,pricewhelan15,laporte16}.
We summarize this observational and theoretical work in Section 2 and 3 respectively, adding in the context of contemporary work from other groups, as well as the larger context of possible connections across the Galactic disk.
Armed with this understanding of the nature of these substructures, we go on in Section 4 to discuss prospects for mapping such structures more generally around our own and other Galaxies.
We end in Section 5 by outlining the motivation for making such maps, asking what they might be telling us about bigger questions: galaxy formation scenarios and the distribution of dark matter around galaxies.

\section{The Nature of Structures Around the Outer Disk --- Summary of Observations}
\label{sec:obs}

From clustering in positions or distance alone, many candidate groups and
over-densities of M giants have been identified in the outer disk or inner halo.
Over the last five years our group has obtained spectroscopy for
candidate members of these structures with the aims of (1) confirming the
existence of substructure in velocities, (2) measuring chemical abundances, and
(3) studying the constituent stellar populations.
These goals then inform our own efforts to produce plausible dynamical
formation scenarios using simulations.
In particular, we {\it avoid} the collimated stellar streams that have been
well-studied in prior work \cite[such as Sgr, Orphan, GD1 and Pal 5 --- see,
e.g.,][]{law10,koposov10,kuepper15,bovy16} and instead focus on groups that
appear as diffuse, amorphous, extended stellar structures such as TriAnd, A13,
and Mon/GASS.
Initial interpretations of these morphologies suggested the structures could be
{\it shells} --- debris from the disruption of satellite galaxies on near-radial
orbits \cite{johnston08} --- as seen from an internal perspective.
However, our own and other recent observations instead suggest a common origin
within the Galactic disk for stars associated with the low-latitude structures.

\subsection{Low-resolution Spectroscopy: Metallicities and Radial Velocities}

In our first results, we extended a prior sample of TriAnd M giants
\cite{rochapinto04} by obtaining spectra of all candidate M giants identified
by applying color-magnitude cuts to stars in the TriAnd region of the sky
\cite{sheffield14}.
We identified M giants associated with the two proposed MSTO TriAnd structures
\cite[TriAnd 1 and 2, as named by][]{martin07,martin14}.
M giant stars in both TriAnd 1 and 2 form clear over-densities in radial
velocities with a small dispersion, $\sigma_v \approx 25~\kms$, compared to the
background halo velocity distribution.
The radial velocities of M giants in both TriAnd 1 and 2 follow the same
sequence in velocity with a steady negative gradient of mean Galactic
standard-of-rest (GSR) radial velocity ($v_{\rm GSR}$) with increasing Galactic
longitude, $l$; see Figures~\ref{fig:ting_vel}--\ref{fig:apw}, purple triangles.
We initially presented a dynamical model that simultaneously and approximately
reproduces TriAnd 1 and 2 as tidal debris stripped over two separate pericentric
passages from a single accreted satellite on a low-eccentricity, retrograde,
near-planar orbit.
The debris structures in the simulation were morphologically closer to {\it
streams} rather than {\it shells}, but still subtended large areas on the sky
as observed from the Sun's position.
We note, however, that because of large distance uncertainties, the M giants in
TriAnd 1 and 2 are indistinguishable and overlap in distance, velocity
distribution, and sky position; the existence of two distinct structures rather
than a single extended structure has yet to be conclusively demonstrated.
Hereafter, we therefore refer to the TriAnd structures collectively, rather
than individually.

In subsequent work, we continued this spectroscopic survey by observing M giant
stars in A13 \cite{sharma10}.
A13 overlaps the TriAnd clouds in Galactic longitude at one end, and GASS at
the other end, but is apparent in the Northern Galactic hemisphere at slightly
brighter magnitudes than TriAnd.
The spectra show that, like the TriAnd clouds, this group has a
coherent velocity structure with low dispersion and a steady gradient with
longitude, $l$, confirming the genuine association of its members \cite{li17};
see Figure~\ref{fig:ting_vel}, orange squares.

Low-resolution spectroscopy has also been obtained for a sample of M giant stars
that span $\approx 100^\circ$ of the Mon/GASS structure \cite{crane03}.
The candidate Mon/GASS member M giants also show a clear trend in GSR velocity
with Galactic longitude, and appear to form a coherent sequence with both A13
and TriAnd; see Figure~\ref{fig:ting_vel}, green circles.

\begin{figure}[!ht]
\centering
\includegraphics[width=5 in]{figures/vgsr.pdf}
\caption{\label{fig:ting_vel}
Summary of the velocity distribution of M giants in each of the three
low-latitude structures.
Markers represent individual stars identified as likely members of each of the
three structures discussed in this work (see figure legend).
Grey curves show circular orbits in the Galactic disk midplane with circular
velocity equal to $220~\kms$ at several Galactocentric cylindrical radii, $R$,
as indicated on the figure.
Velocity uncertainties are typically the same as or smaller than the marker
sizes.
}
\end{figure}

As mentioned above, Figure~\ref{fig:ting_vel} \cite[reproduced with data from
previous work,][]{li17}, summarizes the line-of-sight (GSR) velocity trends of M
giants in each of the three low-latitude structures.
Not only do these groups all have low dispersions relative to the stellar halo
--- which suggests that the groups themselves are real --- they also appear to
{\it collectively} exhibit a continuous gradient with Galactic longitude, $l$.
This suggests that the groups may also be associated with one another, or part
of a larger structure of the outer Galactic disk.

%{\it Summary: the low latitude structures each have low velocity dispersions supporting the genuine association of the stars within them; they also form velocity sequences as functions of Galactic longitude that are continuous across the groups.}

\subsection{Stellar Populations and Other Kinematic Tracers}
 \label{sec:populations}
Motivated by the observed, low-dispersion velocity distribution of the M giant
stars in TriAnd, we sought to observe other distance tracer stars in the same
region, determine their membership, and improve the distance estimates to the
structure.
We focused on and selected RR Lyrae stars in the TriAnd region from the Palomar
Transient Factory \cite[PTF;][]{ptf}, using a conservative distance cut to
account for uncertainties in the RR Lyrae and M giant photometric distance
estimates, $15~\kpc < d_\odot < 35~\kpc$.
We obtained spectra for $\approx$1/3 of the total number of RR Lyrae in the M
giant volume considered to be associated with TriAnd and measured radial
velocities for these stars \cite{pricewhelan15}.

\begin{figure}[t]
\centering
\includegraphics[width=5 in]{figures/triand_rrlyrae}
\caption{\label{fig:apw}
Comparison of the velocity distribution for M giants in the TriAnd structure
(purple triangles) with velocities for RR Lyrae stars in the same region of sky
and distance range (blue circles).
Radial velocity uncertainties of the M giant stars are typically the size of the
marker or smaller.
Uncertainties for the RR Lyrae stars are shown with gray error bars.
Note the low-dispersion sequence in the M giant velocities, unseen in the RR
Lyrae star velocities, which look like typical halo stars with a large
velocity dispersion.
}
\end{figure}

Figure~\ref{fig:apw} \cite[reproduced from][]{pricewhelan15} shows the results
of our survey: unlike the M giants (triangles) the RR Lyrae stars (circles) show
no clear, tight velocity sequence.
By modeling both the RR Lyrae and M giants velocities as having been drawn from
a mixture of two populations --- one representing a low-dispersion foreground
sequence with varied dispersion, and one representing a background halo
population with large dispersion, both Gaussian --- we showed that, after
accounting for selection effects, the number ratio of RR Lyrae to M giants, \frrmg, within the overdensity is $\frrmg < 0.38$ with 95\% confidence.

%\begin{figure}[t]
%\centering
%\includegraphics[width=3 cm]{logo-mdpi}
%\caption{\label{fig:isochrones}
%Isochrones for various ages metallicities}
%\end{figure}

Since we were unable to find any RR Lyrae clearly associated with TriAnd, our
attempt to measure more accurate distances to the structure was unsuccessful.
However, the upper limit on the value of $\frrmg$ was in itself interesting.
RR Lyrae and M giants are tracers of populations with quite distinct
metallicities: stars in the horizontal branch phase of evolution are typically
only blue enough to cross the instability strip and become RR Lyrae if they have
$\feh \lesssim -1.5$; and giant stars typically only evolve to colors red enough
to become spectral class M if they have $\feh \gtrsim -1.5$.
Hence, a stellar population has to contain a significant range of metallicities
in order to contain both types of stars.
%In order to aid with the interpretation of this discovery, Figure~\ref{fig:isochrones} plots isochrones for two $10~{\rm Gyr}$-old populations of low and intermediate metallicity, including the color range for RR Lyrae and M spectral class.
%It emphasizes our understanding that these two types of stars are tracers of populations with quite distinct metallicities and illustrates why, while the initial aim of our survey to find more accurate distances to TriAnd 1 and 2 had failed, our results could be even more intriguing.
The metallicity distributions in nearly all existing satellite galaxies
orbiting the Milky Way \cite[e.g.,][]{kirby11} are typically biased towards
low metallicities, such that they contain no or few M giant stars (i.e. $\frrmg = \infty$).
The largest satellites (the Large Magellanic Cloud and the Sagittarius dwarf galaxy) are exceptions as they contain the most metal rich populations; these dwarfs have $\frrmg \sim 0.5$ \cite{pricewhelan15}.
In contrast, the Galactic disk is an overall metal-rich population and thus has very few RR Lyrae \cite[i.e. $\frrmg \sim 0$;][]{amrose01}, more consistent with our findings for the stellar population of TriAnd.

Our work on the stellar populations of the TriAnd region motivated us to look at possible associations of RR Lyrae with the M giant sequences found in Mon/GASS and A13.
For the surveyed regions of these structures, we find \frrmg\ values similar to those observed for TriAnd, and therefore consistent with membership of the Galactic disk (Sheffield et al., in prep.).

%{\it Summary: the stellar populations in these groups look like each other; they are more consistent with those in the Galactic disk rather than those observed in the stellar halo or Galactic  satellites.}

\subsection{High-resolution Spectroscopy: Abundance Patterns}

\label{sec:abundances}

The origin of stellar associations can also be explored through measurements of
the detailed chemical abundances of their constituent stars.
Interestingly, our finding that the low-latitude structures all have stellar
populations (as indicated by \frrmg) that look more like the disk than known
Galactic satellites (Section~\ref{sec:populations}) appears to be at odds with
prior work on abundance patterns of stars in these structures.
High-resolution spectra of 21 M giants in Mon/GASS \cite{chou2010b} showed
[Ti/Fe] lower by up to 0.4 dex compared to the mean trends known for
main-sequence stars of the Galactic disk \cite[e.g.,][]{reddy03,bensby2014} and
a mean offset for [Y/Fe] of about 0.2 dex at ${\rm [Fe/H]} \approx -0.5$.
A comparison with similar results for stars in the Sgr dwarf spheroidal galaxy
\cite{chou2010a} suggested that Mon/GASS may be more similar to Sgr than the
disk, and therefore proposed an external origin for this structure. Subsequent
work comes to similar conclusions for TriAnd stars \cite{chou11}.

%It should be noted, however, that their results do not agree with the abundances of Ti and, in particular, La, measured in the Canis Major overdensity by \cite{sbordone2005}. Their analysis was based on spectra covering a limited spectral region in the near-IR, including 11 neutral iron and 2 neutral titanium lines. Also \cite{chou2010b} caution that the effects of non-local thermodynamic equilibrium (non-LTE) may be significant, since their measurement of metallicity and Ti abundances relies on the LTE analysis of lines of neutral species,

% ALLY COMMENT: Was this the actual motivation, if we look back at the
% proposal? We were asking for time to explore the possibility of TriAnd being
% a kicked out disk population
In order to explore the apparent contradiction between the stellar populations
and abundance work, we have recently obtained high-resolution ($R \sim
30,000$--$40,000$) and high signal-to-noise ($>200$) spectra of 15 stars in the
TriAnd and A13 overdensities.
14 stars were observed with the HIRES-S spectrograph at the Keck-1 telescope
\cite{vogt1994} and 1 star was observed using the the UVES spectrograph at the
VLT (Program ID: 097.B-0770A).
%The spectral resolving power R of the HIRES spectra is 36\,000 and the UVES data have R $\sim$ 47\,000. All Keck spectra cover the full optical region, from $4800$ to $8770$ $\AA$, although the exact wavelength coverage varies as several slightly different instrument configurations were tried in an effort to get all the key lines into a single exposure. The signal-to-noise (SNR)/\AA of the HIRES spectra near 5200 \AA in the continuum at the center of the echelle order exceeds 200. For the UVES spectrum, the SNR of 50 near 5500 \AA was achieved. The fundamental parameters and chemical abundances of stars were determined using 2MASS and APASS photometry and the high-resolution spectra. All 15 stars are M-type giants with effective temperatures Teff ~ 3800 K and surface gravities log(g) ~ 1 dex. The $T_{\rm eff}$ estimates were derived using the Infrared Flux Method (Casagrande et al. 2010, 2014). Chemical abundances were measured for 6 chemical elements, including O, Na, Mg, Ca, Fe, Eu using LTE and the standard MARCS stellar model atmospheres (Gustafsson et al. 2008). We have also performed detailed analysis of the effects of non-local thermodynamic equilibrium (non-LTE) for the elements, where detailed calculations are available \cite{bergemann2011, bergemann2012, bergemann2016}, however, the NLTE corrections for the chosen lines are minor and do not change our conclusions. In what follows, we use our LTE results, because all studies of chemical abundances in dSph systems and most studies in the Milky Way to date have employed LTE with 1D hydrostatic models
We find that the stars in both structures have extremely similar chemical
abundances, with the abundance dispersion $\leq 0.06$ dex for most chemical
elements.
The TriAnd and A13 stars also have a very narrow metallicity spread, ${\rm
[Fe/H]} = -0.92 \pm 0.06$ dex.
The abundances of all measured alpha-elements are uniformly enhanced at the
level of [O/Fe] = $0.43 \pm 0.06$ dex, [Mg/Fe] = $0.41 \pm 0.02$ dex,
[Na/Fe] = $0.35 \pm 0.06$ dex, and [Eu/Fe] = $0.15 \pm 0.03$ dex.
At this level, the TriAnd and A13 stars are consistent with the abundances of
the in-situ formed Milky Way thick disk
stars\cite{fuhrmann2004,bergemann2014,bensby2014}.
The only exception is [Ca/Fe], which is consistent with solar with the dispersion of 0.06 dex; however, the non-local thermodynamic equilibrium (NLTE) abundance correction for the \ion{Ca}{i} lines for M giants is 0.1 dex that will bring the results in agreement with the MW disk studies that are based on FGK dwarfs, for which the NLTE correction is much smaller \cite{Merle2011}.
The abundance ratios are too high for the chemical abundance patterns observed in the stars of the Galactic satellites \cite{Bonifacio:2000,shetrone2001,shetrone2003,Tolstoy:2009,deBoer:2014}, which are known to have low, typically solar ([O/Fe], [Mg/Fe]) or even sub-solar ([Na/Fe]), ratios at [Fe/H] $\sim -1$.

%For example, Fornax, a dSph galaxy most close to TriAnd 1 and A13 in metallicity, has [Na/Fe] $\sim -0.6$ dex and [Mg/Fe] $\sim 0$ at [Fe/H] $=-1$ (Letarte et al. 2010, Kirby 2010, Lemasle et al. 2014) \footnote{Note that to avoid systematic differences between abundance measurements obtained from spectra taken with different instruments, we have chosen here to compare with the data taken with the same instrument; here UVES and FLAMES at the VLT, ESO, or with HIRES at Keck.}.

%Our results for TriAnd and A13 do not confirm the conclusions by \cite{chou2010b}. Although there could be several astrophysical reasons for the differences, this could be caused by the
One explanation for the disagreement between our and prior abundance work for
stars in the same structures is differences in the observed datasets and
spectroscopic modeling techniques.
%Our observed data cover the full optical range from the near-UV to the near-IR, whereas
Past work \cite{chou2010b,chou11} analysed a small wavelength region in the near-IR, limited to 150 \AA\ from 7440 to 7590 \AA. Because of this limitation, they could include only 11 \ion{Fe}{I} and 2 \ion{Ti}{I} lines in the determination of gravities, metallicities, and abundances. Our analysis, through a much wider wavelength coverage and high SNR attained for the observed spectra, allowed us to include 70 lines of \ion{Fe}{I} and \ion{Fe}{II} \cite{bergemann2012}, as well as 10 lines of \ion{Ti}{2}. Moreover, it has been shown that \ion{Ti}{1} should be not be used in abundance studies because it is very sensitive to NLTE effects \cite{bergemann2011}.
In order to explore the sensitivity of abundance diagnostics to the line selection and wavelength regimes we performed test computations on our own data, using a reduced line-list. We have found that using the line-list from \cite{chou2010a}, the metallicities are over-estimated by $\sim 0.25$ dex and [Ti/Fe] are under-estimated by $0.4$ dex, compared to the results using the complete line-list. This suggests that the choice of the line-list and diagnostic spectral band (full optical or near-IR) as a plausible explanation  of the apparently discrepant results.
%{\it Summary: the abundance patterns  of the low-latitude structures are similar to those of the thick disk of our Galaxy.}

Overall, comparing our work to abundances of disk stars and satellite populations obtained using analogous data sets and reduction techniques, our results suggest that the birthplace of stars in TriAnd and A13 was the outer Galactic disk rather than an infalling satellite galaxy (Bergemann et al., in prep.).

%it is unlikely that the TriAnd 1 and A13 stars originate from a disrupted dSph galaxy

\subsection{Mapping Main-sequence Stars in the Low-latitude Structures}

While we have concentrated on follow-up studies of the known low-latitude structures as traced by M giants selected from 2MASS, knowledge of the spatial distribution of MSTO stars towards the anticenter region has been further refined using photometry from the SDSS \cite{xu15} and Pan-STARRS1 surveys (Lurie et al., in prep.).
Both of these studies employ the novel technique of subtracting color-magnitude diagrams (CMD's) derived from fields in their photometric data which were symmetrically placed at equal and opposite Galactic latitudes and at the same Galactic longitudes.
These differenced CMD's allowed denser regions, closer to the Galactic plane and at smaller heliocentric distances  to be explored.
Both studies showed overdense arcs or stars oscillating between the northern and southern hemispheres as the distance from the Sun was increased towards the anticenter of our Galaxy.
The vast numbers of stars in these surveys allowed more clear identification of these global structures as clearly distinct, separate features.

%{\it Summary: the overdensities around the outer Galaxy, oscillating between the northern and southern hemispheres have been traced to smaller-scale oscillations all the way to the Solar Neighborhood.}

\subsection{Connecting the Low-latitude Structures to Velocity Structure Near the Solar Radius}

% APW COMMENT: what exactly do we mean by "local"?
Coincident with studies exploring structures at the very outer edge of our
Galactic disk, large scale spectroscopic surveys have allowed a detailed
re-examination of the {\it local} distribution of stellar velocities.
Using data from the SDSS \cite{widrow12,yanny13} and the RAdial Velocity
Experiment \cite[RAVE;][]{rave,williams13}, asymmetries between the northern and
southern Galactic hemispheres have been seen in the density and velocity
distributions of stars in the vicinity of the sun.
Looking $\sim$2 kpc out towards the Galactic anticenter, the Large Sky Area Multi-Object Fiber Spectroscopic Telescope \cite[LAMOST;][]{cui12,deng12,zhao12} finds similar asymmetries in radial and vertical velocities \cite{carlin13}.
The scale and sense of these asymmetries indicate moderate systematic motions (of order a few km/s) of stars within the disk perpendicular to the plane, suggesting both vertical movement of the midplane, and compression and rarefaction of the vertical scale \cite[referred to as ``bending'' and ``breathing'' modes respectively --- see, e.g.,][]{widrow14}.
It is natural to associate these asymmetries in motion as a local manifestations of the oscillations traced in space over much larger scales \cite{xu15,pricewhelan15}.
%{\it Summary: small-scale, systematic vertical motions of and within the disk have been detected in the Solar Neighborhood.}


\section{The Nature of Structures Around the Outer Disk --- Summary of Theoretical Interpretations}
\label{sec:theory}

% APW COMMENT: I think Chervin's simulation may be fine as our "cartoon" ... may
% be that the structures are too complex to simplify into a 1D cartoon

% \begin{figure}[t]
% \centering
% \includegraphics[width=4 in]{figures/cartoon.pdf}
% \caption{\label{fig:cartoon}
% TODO: Cartoon of oscillations in space and velocity}
% \end{figure}

% TING COMMENT: everything above line 251 is really summary of observation.
% Maybe this should go to Section 2.6? It is kinda confusing that this section
% is for theory but it were all about observations for the first 20 lines
% Figure \ref{fig:cartoon} is a cartoon which summarizes all the observational works and their implications, by showing the approximate locations and amplitudes of the spatial and velocity structures that have been identified. (Note that the figure is misleading as the structures are {\it not} consistent morphologically with concentric rings that are axisymmetric about the Galactic center.)

The observations summarized in Section \ref{sec:obs} indicate that:
\begin{itemize}
\item the low-latitude structures --- Mon/GASS, TriAnd, and A13  --- each have
      low velocity dispersions supporting the genuine association of the
      candidate member stars;
\item Mon/GASS, TriAnd, and A13 share a continuous sequence in mean GSR velocity
      as a function of Galactic longitude, suggestive of associations between
      these groups;
\item the stellar populations in the structures (as indicated by \frrmg) are all
      more consistent with those in the Galactic disk rather than those observed
      in the stellar halo or Galactic satellites;
\item the abundance patterns of stars in TriAnd and A13 are similar to those
      found in the thick disk of our Galaxy;
\item the low-latitude structures (around the outer disk) may be connected to
      oscillating density and velocity structure on smaller scales, traced all
      the way back to the solar neighborhood;
\end{itemize}
Taken together, we conclude that: (i) there is compelling evidence that
Mon/GASS, TriAnd, and A13 represent populations of stars formed in the disk that
now exist at extreme radii and heights above the Galactic disk;
(ii) these structures are associated and part of a global system of vertical
disk oscillations that can be traced all the way to the velocity asymmetries
seen in the solar neighborhood; and
(iii) the stellar populations in and chemical abundances of these structures are
inconsistent with a picture in which they formed from an accreted satellite.

One natural interpretation of these collected observations is that the oscillations represent the response of the disk to an external perturbation, for example as the impact of a satellite galaxy is transmitted and amplified by its wake in the dark matter halo \cite[as described for the LMC in][]{weinberg06}
Prior work has already pointed to this as a possible explanation for the existence of Mon/GASS \cite{kazantzidis08,younger08}, with the Sagittarius dwarf galaxy being pointed to as a plausible culprit for the perturbation \cite{purcell11}.
It has also been demonstrated how perturbations from a satellite on an orbit perpendicular to the disk could lead to bending (at low relative impact velocity) and breathing (at higher relative velocity) modes that would be observed in the solar neighborhood as asymmetries in the local velocity distribution  \cite{widrow14} and on larger scales as rings \cite{donghia16}.
Simulations have also shown that Sgr could be responsible for local velocity structure  \cite{gomez13}.
Such interactions and corresponding disk features have been found to naturally occur in cosmological simulations \cite{gomez16}.

\begin{figure}[ht!]
\centering
\includegraphics[width=5 in]{figures/simulation.pdf}
\caption{\label{fig:chervin}
% APW COMMENT: needs a caption - waiting until we have final version of figure.
}
\end{figure}

Figure \ref{fig:chervin} illustrates these ideas with the results of simulations from our own recent work.
Using simulations of a disk disturbed (separately) by satellites on orbits that
mimic those expected for the Large Magellanic Cloud and Sgr \cite{laporte17a}, we
extend the prior theoretical backdrop that looked at Mon/GASS to examine whether
the extreme locations of TriAnd stars could fit within the same picture.
With different masses and orbits (and consequently different interaction
strength, timings, and durations) these satellites necessarily induce distinct
but overlapping signatures on the global structure of the disk.
In more recent work, we found that a model that was capable of reproducing the scales of the observed disturbances (radial wavelength and amplitude in space, as well as magnitude of offsets in velocity locally) required:
the interaction of Sgr with the disk of the Milky Way to be followed for several passages longer than prior work;
Sgr must have sufficient initial mass and density to impact the disk in the last Gigayear with a remaining mass of $\sim3\times10^{9}~\msun$;
and the disk to be realized with stars existing as far out as $40~\kpc$ from the Galactic center in order to populate the regions corresponding to TriAnd.
The interaction with the LMC modified the overall morphology of the structures induced, but was not sufficient alone to explain their properties.
The full details of these results will be discussed in an upcoming paper (Laporte et al., in prep.).

\section{Discussion --- Observational Prospects}

\subsection{The Milky Way}

%It is interesting to place the current work in the context of ongoing and near-future studies of our Galaxy.
While the connections that have already been made between the different
low-latitude structures and the disk population are convincing, there are
several possible directions for further observations to strengthen these claims.
More detailed kinematic information and density measurements for more tracers
would greatly facilitate an informative comparison to theoretical work, allowing
more detailed interpretations of the history of our Galaxy.

The most obvious direction is to obtain proper motions and more accurate
distance estimates to the known features, using the candidate members discussed
in this work to search for other tracers.
For example, proper motions for the ``Anti-Center Stream'' (ACS) \cite[which may
or may not be part of the larger Mon/GASS structure;] []{li12} indicate that
stars in the ACS are not actually moving parallel to the stream \cite{carlin10}.
This is inconsistent with expectations for the behavior of debris from a
destroyed satellite.
If similar measurements of proper motions of stars in all of the low-latitude
structures showed significant motion perpendicular to the Galactic disk, this
would provide conclusive evidence of a disk origin and connection to local
oscillations.
With precise proper motions, the velocity information would also place important
constraints on dynamical models of the disk (see Section~\ref{sec:conc}).
In upcoming data-releases, astrometric data from the {\it Gaia} mission
\cite{gaia} is poised to provide these data.
Expected proper motion uncertainties for the closest M giant stars ($\approx
5~\kpc$) in Mon/GASS correspond to tangential velocity uncertainties of $\approx
1$--$2~\kms$ for M giants with tangential velocities $<50~\kms$.
For the farthest known M giant stars in TriAnd ($\approx 35~\kpc$), tangential
velocity uncertainties will be $\approx 7$--$12~\kms$ for M giants with
tangential velocities $<50~\kms$.

Another complimentary direction for future data is to extend the photometric and spectroscopic maps to fainter magnitudes and global scales.
{\it Gaia} will be able to tackle this with distances, proper motions, and
radial velocities, although its reach towards low latitudes in the inner Galaxy
will be limited by extinction.
Infrared surveys, such as APOGEE \cite{apogee}, could overcome this limitation
and reach entirely across the Galaxy. Spectra from such a survey could provide
both radial velocities and abundance patterns.

% APW COMMENT: you mention LSST in next section, but what about how LSST will
% provide deep photometry - map far down the main sequence for stars in these
% structures.

% APW COMMENT: I feel like we need to say more about APOGEE, future
% spectroscopic surveys like DESI, PFS.

\subsection{Other Galaxies}

% TING COMMENT: There are also a significant amount of studies on "galaxy
% morphology and warps" with distance galaxies. For example:
% http://adsabs.harvard.edu/abs/2016JKAS...49..239A
% You may want to mention these studies (reference therein) as well....

% APW COMMENT: this sentence comes a bit out of the blue -- needs a better
% introductory or transitional statement
The great advantage of star-count studies is the ability to reach extremely low
surface brightnesses. For example, TriAnd is estimated to have a surface
brightness lower than $\Sigma <$32 mag/arcsec$^2$ \cite{majewski04}.
%For example, TriAnd I contains ?? M-giant stars spread over an area on the sky of roughly ????, so ? giants/deg$^2$.
%Adopting a 10 Gyr-old isochrone for this population \cite[from the Padova group][]{}, each M-giant has an associated total stellar luminosity of ??? in the ?? band. Hence, the equivalanet urfc ebright ness would be .....

% APW COMMENT: this section needs some work, but ran out of steam...

The growing samples of galaxies within and beyond the Local Group being mapped to extremely low surface brightness levels are intriguing.
For nearby galaxies, these are through star counts studies, most spectacularly for the case of our nearest neighbor, the Andromeda Galaxy, where giant star counts have revealed a significantly extended and richly featured outer stellar disk \cite{ferguson02,ibata05}.
Analogous studies have been carried out for galaxies up to distances of several Mpc \cite[e.g.,][]{monachesi13,crnojevi16}, although the focus of these studies has typically been on detecting the stellar halo of these objects.
Several dedicated surveys have made innovative advances in studying galaxies to low surface-brightness using a variety of techniques to reach limits below 30 mag/arcsec$^2$ in integrated light \citep[e.g.][]{delgado10,vandokkum14,duc15}.

Looking to the future,
NASA's proposed WFIRST satellite, with its wide field of view and high resolution, offers the possibility of extending the deep star-count sensitivities now achieved in MW and M31 to all galaxies within 10 Mpc \cite{spergel13}; and
the next decade will also see first light for the Large Synoptic Survey Telescope (LSST), from which images can be combined to be sensitive to slightly shallower depth \citep[$\sim$29 mag/arcsec$^2$, see][]{ivezic08} but for vast numbers (many millions!) of galaxies.
% TING COMMENT: about 29 mag/arcsec - I tried to find the number "29
% mag/arcsec^2" in the reference but I could not. I doubt this number is true
% if you are talking about stellar resolved overdensities. TriAnd is 32 mag/
% arcsec^2 and can be discovered with a much smaller telescope. Maybe you are
% talking about galaxies (star not resolved) only... But it is kinda confusing


\section{Conclusion --- What Might These Structures Tell Us About Galaxies?}

%{\it KVJ --- Reminder to self to add these somewhere} \\
%arXiv:1706.01900  --- Title: Milky Way Tomography with K and M Dwarf Stars: the Vertical Structure of the Galactic Disk \\
 %D'Onghia et al 2016 on sat and disk interaction \\
% Bovy et al 2015 - power spectrum of vel in disk \\
% Kazantzidis08 - rings etc \\
% schwarzkopf 01 \\
 %Zarik=tsky 97 - lopsided gals and accreiton

\label{sec:conc}

The above sections summarize observational evidence for large scale vertical
oscillations of the Galactic plane present in the solar neighborhood and
reaching out beyond the traditional edge of our stellar disk.
We have also discussed theoretical studies that suggest that these oscillations
could be caused by, and contain the signatures of, ongoing interactions of the
Milky Way with its satellite system.
Moreover, there are numerous observational prospects for extending this work
both to globally map the Milky Way and to look for analogous features around
many other galaxies.

Now that we have a physical picture of the origin of such features, as well as prospects for mapping them further within the Milky Way and detecting analogous substructures around other galaxies, we can move on to discussing how useful they are for constraining the dynamics and evolution of galaxies.
While the mere existence of these structures is interesting, they contain a tiny fraction of the stars in galaxies spread out over a large area --- these properties naturally make them difficult to map, either because their unique signatures can be lost in the foreground star counts (e.g., in the Milky Way), or because the required surface brightness limits for detection are prohibitively low (for integrated light).

Conversely, these features around the outskirts of galaxies may prove to be particularly powerful probes of interactions and histories, precisely because they contain so little mass: they can be modeled as test particles responding to an external perturbation.

Below are just three examples of where these structures could promise new insights into some classic questions.
\begin{description}
\item{\it Disk heating mechanisms ---}
It has been understood for a long time that disks can evolve significantly due to mergers, major or minor, and hence that their current structures bear witness to their accretion history \cite{toth92,quinn93,walker96,velazquez99}.
This understanding has fueled a significant literature on the importance of the heating of galactic disks in response to encounters with other dark matter halos (that may or may not contain their own galaxies) \cite{font01,ardi03,benson04,stewart08,hopkins08,villalobos08,purcell09,kazantzidis09,sachdeva16,moetazedian16}.
In general, these studies have concentrated on the overall effects of many encounters on global properties, such as the thickness and vertical velocity dispersion in disks.
Their results have traditionally been compared to these scales in samples of galaxies.
In contrast, the identification and mapping of vertical waves associated with ongoing interactions in the Milky Way gives us the opportunity to dissect individual disk heating events in progress.
We can use this to check our understanding of the mechanism directly and in detail rather than assessing its importance through collective effects and longterm, phase-mixed signatures.
\item{\it Stellar halo formation processes ----}
The last decade has seen increasing interest in assessing how much of the content of stellar halos could be made from stars originally formed in the disks of the galaxies that they surround. Hydrodynamical simulations of galaxy formation suggest that tens of percent of the inner halo might be formed this way
\cite{abadi06,zolotov09,zolotov10,font11,mccarthy12,tissera13,tissera14,pillepich15,cooper15}.
Preliminary arguments for the existence of this ``kicked-out-disk'' population were based on transitions in the density or orbital structures of stellar halos \cite[e.g.,][]{carollo07}.
However, such transitions were also found to naturally occur in purely-accreted models of stellar halos \cite{deason13}.
More convincing observational evidence for disk stars in the halo is just beginning to emerge through studies that look for stars with halo-like kinematics, but disk-like abundances around M31 \cite{dorman13} and the Milky Way \cite{sheffield12,hawkins15,bonaca17}.
Our own work adds new perspectives on this stellar halo formation process with the detection and modeling of disk stars that may be in transition from the disk to the halo.
\item{\it Galactoseismic probes of interactions and dark matter ---}
The response of a disk to an encounter will depend on its own properties, the properties of the dark matter halo in which it is embedded and the mass and orbit of the perturbing satellite.
This leads to the suggestion that, analogous to helioseismic investigations of the structure of our Sun, maps of a disk response --- such as those described in Section \ref{sec:obs} for our Milky Way --- might be similarly inverted to tell us about (e.g.) the structure of the dark matter halo
\cite{widrow12}.
Indeed, recent studies of the signatures of encounters in the very outskirts of extended HI disks have successfully used simulations combined with an analytic understanding to find how the observed characteristics of the disturbed gas can be simply related to properties of the perturbing object \cite{chakrabarti09,chakrabarti11b,chang11}.
\end{description}

Uplifting/Concluding sentence.

%%%%%%%%%%%%%%%%%%%%%%%%%%%%%%%%%%%%%%%%%%
\vspace{6pt}

%%%%%%%%%%%%%%%%%%%%%%%%%%%%%%%%%%%%%%%%%%
%% optional
%\supplementary{The following are available online at www.mdpi.com/link, Figure S1: title, Table S1: title, Video S1: title.}

%%%%%%%%%%%%%%%%%%%%%%%%%%%%%%%%%%%%%%%%%%
\acknowledgments{Much of the work reviewed in this paper was made possible by NSF grants AST-1312196.
K.V.J. was supported by NSF grant AST-1614743 while writing the review.}

%%%%%%%%%%%%%%%%%%%%%%%%%%%%%%%%%%%%%%%%%%
\authorcontributions{This paper summarizes results from the team of listed authors. It was written by K.V.J and A.P.W., with section contributions from M.B.. Figures were made by Adrian Price-Whelan and C.L.. All the authors reviewed and commented on the drafts.}

%%%%%%%%%%%%%%%%%%%%%%%%%%%%%%%%%%%%%%%%%%
\conflictsofinterest{The authors declare no conflict of interest.}

%%%%%%%%%%%%%%%%%%%%%%%%%%%%%%%%%%%%%%%%%%
%% optional
%\abbreviations{The following abbreviations are used in this manuscript:\\

%\noindent
%\begin{tabular}{@{}ll}
%MDPI & Multidisciplinary Digital Publishing Institute\\
%DOAJ & Directory of open access journals\\
%TLA & Three letter acronym\\
%LD & linear dichroism
%\end{tabular}}

%%%%%%%%%%%%%%%%%%%%%%%%%%%%%%%%%%%%%%%%%%
%%%%%%%%%%%%%%%%%%%%%%%%%%%%%%%%%%%%%%%%%%
% Citations and References in Supplementary files are permitted provided that they also appear in the reference list here.

%=====================================
% References, variant A: internal bibliography
%=====================================
\externalbibliography{yes}
% \bibliographystyle{chicago2}
\bibliography{refs}

% \begin{thebibliography}{-------}
%\providecommand{Natureexlab}[1]{#1}

\bibitem[{ESA}(1997)]{esa97}
{ESA}., Ed.
\newblock {\em {The HIPPARCOS and TYCHO catalogues. Astrometric and photometric
  star catalogues derived from the ESA HIPPARCOS Space Astrometry Mission}},
  Vol. 1200, {\em ESA Special Publication},  1997.

\bibitem[{Dehnen}(1998)]{dehnen98}
{Dehnen}, W.
\newblock {The Distribution of Nearby Stars in Velocity Space Inferred from
  HIPPARCOS Data}.
\newblock {\em AJ} {\bf 1998}, {\em 115},~2384--2396,
  \href{http://xxx.lanl.gov/abs/astro-ph/9803110}{{\normalfont
  [astro-ph/9803110]}}.

\bibitem[{Dehnen}(2000)]{dehnen00}
{Dehnen}, W.
\newblock {The Effect of the Outer Lindblad Resonance of the Galactic Bar on
  the Local Stellar Velocity Distribution}.
\newblock {\em AJ} {\bf 2000}, {\em 119},~800--812,
  \href{http://xxx.lanl.gov/abs/astro-ph/9911161}{{\normalfont
  [astro-ph/9911161]}}.

\bibitem[{York} \em{et~al.}(2000){York}, {Adelman}, {Anderson}, {Anderson},
  {Annis}, {Bahcall}, {Bakken}, {Barkhouser}, {Bastian}, {Berman}, {Boroski},
  {Bracker}, {Briegel}, {Briggs}, {Brinkmann}, {Brunner}, {Burles}, {Carey},
  {Carr}, {Castander}, {Chen}, {Colestock}, {Connolly}, {Crocker}, {Csabai},
  {Czarapata}, {Davis}, {Doi}, {Dombeck}, {Eisenstein}, {Ellman}, {Elms},
  {Evans}, {Fan}, {Federwitz}, {Fiscelli}, {Friedman}, {Frieman}, {Fukugita},
  {Gillespie}, {Gunn}, {Gurbani}, {de Haas}, {Haldeman}, {Harris}, {Hayes},
  {Heckman}, {Hennessy}, {Hindsley}, {Holm}, {Holmgren}, {Huang}, {Hull},
  {Husby}, {Ichikawa}, {Ichikawa}, {Ivezi{\'c}}, {Kent}, {Kim}, {Kinney},
  {Klaene}, {Kleinman}, {Kleinman}, {Knapp}, {Korienek}, {Kron}, {Kunszt},
  {Lamb}, {Lee}, {Leger}, {Limmongkol}, {Lindenmeyer}, {Long}, {Loomis},
  {Loveday}, {Lucinio}, {Lupton}, {MacKinnon}, {Mannery}, {Mantsch}, {Margon},
  {McGehee}, {McKay}, {Meiksin}, {Merelli}, {Monet}, {Munn}, {Narayanan},
  {Nash}, {Neilsen}, {Neswold}, {Newberg}, {Nichol}, {Nicinski}, {Nonino},
  {Okada}, {Okamura}, {Ostriker}, {Owen}, {Pauls}, {Peoples}, {Peterson},
  {Petravick}, {Pier}, {Pope}, {Pordes}, {Prosapio}, {Rechenmacher}, {Quinn},
  {Richards}, {Richmond}, {Rivetta}, {Rockosi}, {Ruthmansdorfer}, {Sandford},
  {Schlegel}, {Schneider}, {Sekiguchi}, {Sergey}, {Shimasaku}, {Siegmund},
  {Smee}, {Smith}, {Snedden}, {Stone}, {Stoughton}, {Strauss}, {Stubbs},
  {SubbaRao}, {Szalay}, {Szapudi}, {Szokoly}, {Thakar}, {Tremonti}, {Tucker},
  {Uomoto}, {Vanden Berk}, {Vogeley}, {Waddell}, {Wang}, {Watanabe},
  {Weinberg}, {Yanny}, {Yasuda}, and {SDSS Collaboration}]{york00}
{York}, D.G.; {Adelman}, J.; {Anderson}, Jr., J.E.; {Anderson}, S.F.; {Annis},
  J.; {Bahcall}, N.A.; {Bakken}, J.A.; {Barkhouser}, R.; {Bastian}, S.;
  {Berman}, E.; {Boroski}, W.N.; {Bracker}, S.; {Briegel}, C.; {Briggs}, J.W.;
  {Brinkmann}, J.; {Brunner}, R.; {Burles}, S.; {Carey}, L.; {Carr}, M.A.;
  {Castander}, F.J.; {Chen}, B.; {Colestock}, P.L.; {Connolly}, A.J.;
  {Crocker}, J.H.; {Csabai}, I.; {Czarapata}, P.C.; {Davis}, J.E.; {Doi}, M.;
  {Dombeck}, T.; {Eisenstein}, D.; {Ellman}, N.; {Elms}, B.R.; {Evans}, M.L.;
  {Fan}, X.; {Federwitz}, G.R.; {Fiscelli}, L.; {Friedman}, S.; {Frieman},
  J.A.; {Fukugita}, M.; {Gillespie}, B.; {Gunn}, J.E.; {Gurbani}, V.K.; {de
  Haas}, E.; {Haldeman}, M.; {Harris}, F.H.; {Hayes}, J.; {Heckman}, T.M.;
  {Hennessy}, G.S.; {Hindsley}, R.B.; {Holm}, S.; {Holmgren}, D.J.; {Huang},
  C.h.; {Hull}, C.; {Husby}, D.; {Ichikawa}, S.I.; {Ichikawa}, T.;
  {Ivezi{\'c}}, {\v Z}.; {Kent}, S.; {Kim}, R.S.J.; {Kinney}, E.; {Klaene}, M.;
  {Kleinman}, A.N.; {Kleinman}, S.; {Knapp}, G.R.; {Korienek}, J.; {Kron},
  R.G.; {Kunszt}, P.Z.; {Lamb}, D.Q.; {Lee}, B.; {Leger}, R.F.; {Limmongkol},
  S.; {Lindenmeyer}, C.; {Long}, D.C.; {Loomis}, C.; {Loveday}, J.; {Lucinio},
  R.; {Lupton}, R.H.; {MacKinnon}, B.; {Mannery}, E.J.; {Mantsch}, P.M.;
  {Margon}, B.; {McGehee}, P.; {McKay}, T.A.; {Meiksin}, A.; {Merelli}, A.;
  {Monet}, D.G.; {Munn}, J.A.; {Narayanan}, V.K.; {Nash}, T.; {Neilsen}, E.;
  {Neswold}, R.; {Newberg}, H.J.; {Nichol}, R.C.; {Nicinski}, T.; {Nonino}, M.;
  {Okada}, N.; {Okamura}, S.; {Ostriker}, J.P.; {Owen}, R.; {Pauls}, A.G.;
  {Peoples}, J.; {Peterson}, R.L.; {Petravick}, D.; {Pier}, J.R.; {Pope}, A.;
  {Pordes}, R.; {Prosapio}, A.; {Rechenmacher}, R.; {Quinn}, T.R.; {Richards},
  G.T.; {Richmond}, M.W.; {Rivetta}, C.H.; {Rockosi}, C.M.; {Ruthmansdorfer},
  K.; {Sandford}, D.; {Schlegel}, D.J.; {Schneider}, D.P.; {Sekiguchi}, M.;
  {Sergey}, G.; {Shimasaku}, K.; {Siegmund}, W.A.; {Smee}, S.; {Smith}, J.A.;
  {Snedden}, S.; {Stone}, R.; {Stoughton}, C.; {Strauss}, M.A.; {Stubbs}, C.;
  {SubbaRao}, M.; {Szalay}, A.S.; {Szapudi}, I.; {Szokoly}, G.P.; {Thakar},
  A.R.; {Tremonti}, C.; {Tucker}, D.L.; {Uomoto}, A.; {Vanden Berk}, D.;
  {Vogeley}, M.S.; {Waddell}, P.; {Wang}, S.i.; {Watanabe}, M.; {Weinberg},
  D.H.; {Yanny}, B.; {Yasuda}, N.; {SDSS Collaboration}.
\newblock {The Sloan Digital Sky Survey: Technical Summary}.
\newblock {\em AJ} {\bf 2000}, {\em 120},~1579--1587,
  \href{http://xxx.lanl.gov/abs/astro-ph/0006396}{{\normalfont
  [astro-ph/0006396]}}.

\bibitem[{Stoughton} \em{et~al.}(2002){Stoughton}, {Lupton}, {Bernardi},
  {Blanton}, {Burles}, {Castander}, {Connolly}, {Eisenstein}, {Frieman},
  {Hennessy}, {Hindsley}, {Ivezi{\'c}}, {Kent}, {Kunszt}, {Lee}, {Meiksin},
  {Munn}, {Newberg}, {Nichol}, {Nicinski}, {Pier}, {Richards}, {Richmond},
  {Schlegel}, {Smith}, {Strauss}, {SubbaRao}, {Szalay}, {Thakar}, {Tucker},
  {Vanden Berk}, {Yanny}, {Adelman}, {Anderson}, {Anderson}, {Annis},
  {Bahcall}, {Bakken}, {Bartelmann}, {Bastian}, {Bauer}, {Berman},
  {B{\"o}hringer}, {Boroski}, {Bracker}, {Briegel}, {Briggs}, {Brinkmann},
  {Brunner}, {Carey}, {Carr}, {Chen}, {Christian}, {Colestock}, {Crocker},
  {Csabai}, {Czarapata}, {Dalcanton}, {Davidsen}, {Davis}, {Dehnen},
  {Dodelson}, {Doi}, {Dombeck}, {Donahue}, {Ellman}, {Elms}, {Evans}, {Eyer},
  {Fan}, {Federwitz}, {Friedman}, {Fukugita}, {Gal}, {Gillespie}, {Glazebrook},
  {Gray}, {Grebel}, {Greenawalt}, {Greene}, {Gunn}, {de Haas}, {Haiman},
  {Haldeman}, {Hall}, {Hamabe}, {Hansen}, {Harris}, {Harris}, {Harvanek},
  {Hawley}, {Hayes}, {Heckman}, {Helmi}, {Henden}, {Hogan}, {Hogg}, {Holmgren},
  {Holtzman}, {Huang}, {Hull}, {Ichikawa}, {Ichikawa}, {Johnston}, {Kauffmann},
  {Kim}, {Kimball}, {Kinney}, {Klaene}, {Kleinman}, {Klypin}, {Knapp},
  {Korienek}, {Krolik}, {Kron}, {Krzesi{\'n}ski}, {Lamb}, {Leger},
  {Limmongkol}, {Lindenmeyer}, {Long}, {Loomis}, {Loveday}, {MacKinnon},
  {Mannery}, {Mantsch}, {Margon}, {McGehee}, {McKay}, {McLean}, {Menou},
  {Merelli}, {Mo}, {Monet}, {Nakamura}, {Narayanan}, {Nash}, {Neilsen},
  {Newman}, {Nitta}, {Odenkirchen}, {Okada}, {Okamura}, {Ostriker}, {Owen},
  {Pauls}, {Peoples}, {Peterson}, {Petravick}, {Pope}, {Pordes}, {Postman},
  {Prosapio}, {Quinn}, {Rechenmacher}, {Rivetta}, {Rix}, {Rockosi}, {Rosner},
  {Ruthmansdorfer}, {Sandford}, {Schneider}, {Scranton}, {Sekiguchi}, {Sergey},
  {Sheth}, {Shimasaku}, {Smee}, {Snedden}, {Stebbins}, {Stubbs}, {Szapudi},
  {Szkody}, {Szokoly}, {Tabachnik}, {Tsvetanov}, {Uomoto}, {Vogeley}, {Voges},
  {Waddell}, {Walterbos}, {Wang}, {Watanabe}, {Weinberg}, {White}, {White},
  {Wilhite}, {Wolfe}, {Yasuda}, {York}, {Zehavi}, and {Zheng}]{stoughton02}
{Stoughton}, C.; {Lupton}, R.H.; {Bernardi}, M.; {Blanton}, M.R.; {Burles}, S.;
  {Castander}, F.J.; {Connolly}, A.J.; {Eisenstein}, D.J.; {Frieman}, J.A.;
  {Hennessy}, G.S.; {Hindsley}, R.B.; {Ivezi{\'c}}, {\v Z}.; {Kent}, S.;
  {Kunszt}, P.Z.; {Lee}, B.C.; {Meiksin}, A.; {Munn}, J.A.; {Newberg}, H.J.;
  {Nichol}, R.C.; {Nicinski}, T.; {Pier}, J.R.; {Richards}, G.T.; {Richmond},
  M.W.; {Schlegel}, D.J.; {Smith}, J.A.; {Strauss}, M.A.; {SubbaRao}, M.;
  {Szalay}, A.S.; {Thakar}, A.R.; {Tucker}, D.L.; {Vanden Berk}, D.E.; {Yanny},
  B.; {Adelman}, J.K.; {Anderson}, Jr., J.E.; {Anderson}, S.F.; {Annis}, J.;
  {Bahcall}, N.A.; {Bakken}, J.A.; {Bartelmann}, M.; {Bastian}, S.; {Bauer},
  A.; {Berman}, E.; {B{\"o}hringer}, H.; {Boroski}, W.N.; {Bracker}, S.;
  {Briegel}, C.; {Briggs}, J.W.; {Brinkmann}, J.; {Brunner}, R.; {Carey}, L.;
  {Carr}, M.A.; {Chen}, B.; {Christian}, D.; {Colestock}, P.L.; {Crocker},
  J.H.; {Csabai}, I.; {Czarapata}, P.C.; {Dalcanton}, J.; {Davidsen}, A.F.;
  {Davis}, J.E.; {Dehnen}, W.; {Dodelson}, S.; {Doi}, M.; {Dombeck}, T.;
  {Donahue}, M.; {Ellman}, N.; {Elms}, B.R.; {Evans}, M.L.; {Eyer}, L.; {Fan},
  X.; {Federwitz}, G.R.; {Friedman}, S.; {Fukugita}, M.; {Gal}, R.;
  {Gillespie}, B.; {Glazebrook}, K.; {Gray}, J.; {Grebel}, E.K.; {Greenawalt},
  B.; {Greene}, G.; {Gunn}, J.E.; {de Haas}, E.; {Haiman}, Z.; {Haldeman}, M.;
  {Hall}, P.B.; {Hamabe}, M.; {Hansen}, B.; {Harris}, F.H.; {Harris}, H.;
  {Harvanek}, M.; {Hawley}, S.L.; {Hayes}, J.J.E.; {Heckman}, T.M.; {Helmi},
  A.; {Henden}, A.; {Hogan}, C.J.; {Hogg}, D.W.; {Holmgren}, D.J.; {Holtzman},
  J.; {Huang}, C.H.; {Hull}, C.; {Ichikawa}, S.I.; {Ichikawa}, T.; {Johnston},
  D.E.; {Kauffmann}, G.; {Kim}, R.S.J.; {Kimball}, T.; {Kinney}, E.; {Klaene},
  M.; {Kleinman}, S.J.; {Klypin}, A.; {Knapp}, G.R.; {Korienek}, J.; {Krolik},
  J.; {Kron}, R.G.; {Krzesi{\'n}ski}, J.; {Lamb}, D.Q.; {Leger}, R.F.;
  {Limmongkol}, S.; {Lindenmeyer}, C.; {Long}, D.C.; {Loomis}, C.; {Loveday},
  J.; {MacKinnon}, B.; {Mannery}, E.J.; {Mantsch}, P.M.; {Margon}, B.;
  {McGehee}, P.; {McKay}, T.A.; {McLean}, B.; {Menou}, K.; {Merelli}, A.; {Mo},
  H.J.; {Monet}, D.G.; {Nakamura}, O.; {Narayanan}, V.K.; {Nash}, T.;
  {Neilsen}, Jr., E.H.; {Newman}, P.R.; {Nitta}, A.; {Odenkirchen}, M.;
  {Okada}, N.; {Okamura}, S.; {Ostriker}, J.P.; {Owen}, R.; {Pauls}, A.G.;
  {Peoples}, J.; {Peterson}, R.S.; {Petravick}, D.; {Pope}, A.; {Pordes}, R.;
  {Postman}, M.; {Prosapio}, A.; {Quinn}, T.R.; {Rechenmacher}, R.; {Rivetta},
  C.H.; {Rix}, H.W.; {Rockosi}, C.M.; {Rosner}, R.; {Ruthmansdorfer}, K.;
  {Sandford}, D.; {Schneider}, D.P.; {Scranton}, R.; {Sekiguchi}, M.; {Sergey},
  G.; {Sheth}, R.; {Shimasaku}, K.; {Smee}, S.; {Snedden}, S.A.; {Stebbins},
  A.; {Stubbs}, C.; {Szapudi}, I.; {Szkody}, P.; {Szokoly}, G.P.; {Tabachnik},
  S.; {Tsvetanov}, Z.; {Uomoto}, A.; {Vogeley}, M.S.; {Voges}, W.; {Waddell},
  P.; {Walterbos}, R.; {Wang}, S.i.; {Watanabe}, M.; {Weinberg}, D.H.; {White},
  R.L.; {White}, S.D.M.; {Wilhite}, B.; {Wolfe}, D.; {Yasuda}, N.; {York},
  D.G.; {Zehavi}, I.; {Zheng}, W.
\newblock {Sloan Digital Sky Survey: Early Data Release}.
\newblock {\em AJ} {\bf 2002}, {\em 123},~485--548.

\bibitem[{Abazajian} \em{et~al.}(2003){Abazajian}, {Adelman-McCarthy},
  {Ag{\"u}eros}, {Allam}, {Anderson}, {Annis}, {Bahcall}, {Baldry}, {Bastian},
  {Berlind}, {Bernardi}, {Blanton}, {Blythe}, {Bochanski}, {Boroski},
  {Brewington}, {Briggs}, {Brinkmann}, {Brunner}, {Budav{\'a}ri}, {Carey},
  {Carr}, {Castander}, {Chiu}, {Collinge}, {Connolly}, {Covey}, {Csabai},
  {Dalcanton}, {Dodelson}, {Doi}, {Dong}, {Eisenstein}, {Evans}, {Fan},
  {Feldman}, {Finkbeiner}, {Friedman}, {Frieman}, {Fukugita}, {Gal},
  {Gillespie}, {Glazebrook}, {Gonzalez}, {Gray}, {Grebel}, {Grodnicki}, {Gunn},
  {Gurbani}, {Hall}, {Hao}, {Harbeck}, {Harris}, {Harris}, {Harvanek},
  {Hawley}, {Heckman}, {Helmboldt}, {Hendry}, {Hennessy}, {Hindsley}, {Hogg},
  {Holmgren}, {Holtzman}, {Homer}, {Hui}, {Ichikawa}, {Ichikawa}, {Inkmann},
  {Ivezi{\'c}}, {Jester}, {Johnston}, {Jordan}, {Jordan}, {Jorgensen},
  {Juri{\'c}}, {Kauffmann}, {Kent}, {Kleinman}, {Knapp}, {Kniazev}, {Kron},
  {Krzesi{\'n}ski}, {Kunszt}, {Kuropatkin}, {Lamb}, {Lampeitl}, {Laubscher},
  {Lee}, {Leger}, {Li}, {Lidz}, {Lin}, {Loh}, {Long}, {Loveday}, {Lupton},
  {Malik}, {Margon}, {McGehee}, {McKay}, {Meiksin}, {Miknaitis}, {Moorthy},
  {Munn}, {Murphy}, {Nakajima}, {Narayanan}, {Nash}, {Neilsen}, {Newberg},
  {Newman}, {Nichol}, {Nicinski}, {Nieto-Santisteban}, {Nitta}, {Odenkirchen},
  {Okamura}, {Ostriker}, {Owen}, {Padmanabhan}, {Peoples}, {Pier}, {Pindor},
  {Pope}, {Quinn}, {Rafikov}, {Raymond}, {Richards}, {Richmond}, {Rix},
  {Rockosi}, {Schaye}, {Schlegel}, {Schneider}, {Schroeder}, {Scranton},
  {Sekiguchi}, {Seljak}, {Sergey}, {Sesar}, {Sheldon}, {Shimasaku}, {Siegmund},
  {Silvestri}, {Sinisgalli}, {Sirko}, {Smith}, {Smol{\v c}i{\'c}}, {Snedden},
  {Stebbins}, {Steinhardt}, {Stinson}, {Stoughton}, {Strateva}, {Strauss},
  {SubbaRao}, {Szalay}, {Szapudi}, {Szkody}, {Tasca}, {Tegmark}, {Thakar},
  {Tremonti}, {Tucker}, {Uomoto}, {Vanden Berk}, {Vandenberg}, {Vogeley},
  {Voges}, {Vogt}, {Walkowicz}, {Weinberg}, {West}, {White}, {Wilhite},
  {Willman}, {Xu}, {Yanny}, {Yarger}, {Yasuda}, {Yip}, {Yocum}, {York},
  {Zakamska}, {Zehavi}, {Zheng}, {Zibetti}, and {Zucker}]{abazajian03}
{Abazajian}, K.; {Adelman-McCarthy}, J.K.; {Ag{\"u}eros}, M.A.; {Allam}, S.S.;
  {Anderson}, S.F.; {Annis}, J.; {Bahcall}, N.A.; {Baldry}, I.K.; {Bastian},
  S.; {Berlind}, A.; {Bernardi}, M.; {Blanton}, M.R.; {Blythe}, N.;
  {Bochanski}, Jr., J.J.; {Boroski}, W.N.; {Brewington}, H.; {Briggs}, J.W.;
  {Brinkmann}, J.; {Brunner}, R.J.; {Budav{\'a}ri}, T.; {Carey}, L.N.; {Carr},
  M.A.; {Castander}, F.J.; {Chiu}, K.; {Collinge}, M.J.; {Connolly}, A.J.;
  {Covey}, K.R.; {Csabai}, I.; {Dalcanton}, J.J.; {Dodelson}, S.; {Doi}, M.;
  {Dong}, F.; {Eisenstein}, D.J.; {Evans}, M.L.; {Fan}, X.; {Feldman}, P.D.;
  {Finkbeiner}, D.P.; {Friedman}, S.D.; {Frieman}, J.A.; {Fukugita}, M.; {Gal},
  R.R.; {Gillespie}, B.; {Glazebrook}, K.; {Gonzalez}, C.F.; {Gray}, J.;
  {Grebel}, E.K.; {Grodnicki}, L.; {Gunn}, J.E.; {Gurbani}, V.K.; {Hall}, P.B.;
  {Hao}, L.; {Harbeck}, D.; {Harris}, F.H.; {Harris}, H.C.; {Harvanek}, M.;
  {Hawley}, S.L.; {Heckman}, T.M.; {Helmboldt}, J.F.; {Hendry}, J.S.;
  {Hennessy}, G.S.; {Hindsley}, R.B.; {Hogg}, D.W.; {Holmgren}, D.J.;
  {Holtzman}, J.A.; {Homer}, L.; {Hui}, L.; {Ichikawa}, S.i.; {Ichikawa}, T.;
  {Inkmann}, J.P.; {Ivezi{\'c}}, {\v Z}.; {Jester}, S.; {Johnston}, D.E.;
  {Jordan}, B.; {Jordan}, W.P.; {Jorgensen}, A.M.; {Juri{\'c}}, M.;
  {Kauffmann}, G.; {Kent}, S.M.; {Kleinman}, S.J.; {Knapp}, G.R.; {Kniazev},
  A.Y.; {Kron}, R.G.; {Krzesi{\'n}ski}, J.; {Kunszt}, P.Z.; {Kuropatkin}, N.;
  {Lamb}, D.Q.; {Lampeitl}, H.; {Laubscher}, B.E.; {Lee}, B.C.; {Leger}, R.F.;
  {Li}, N.; {Lidz}, A.; {Lin}, H.; {Loh}, Y.S.; {Long}, D.C.; {Loveday}, J.;
  {Lupton}, R.H.; {Malik}, T.; {Margon}, B.; {McGehee}, P.M.; {McKay}, T.A.;
  {Meiksin}, A.; {Miknaitis}, G.A.; {Moorthy}, B.K.; {Munn}, J.A.; {Murphy},
  T.; {Nakajima}, R.; {Narayanan}, V.K.; {Nash}, T.; {Neilsen}, Jr., E.H.;
  {Newberg}, H.J.; {Newman}, P.R.; {Nichol}, R.C.; {Nicinski}, T.;
  {Nieto-Santisteban}, M.; {Nitta}, A.; {Odenkirchen}, M.; {Okamura}, S.;
  {Ostriker}, J.P.; {Owen}, R.; {Padmanabhan}, N.; {Peoples}, J.; {Pier}, J.R.;
  {Pindor}, B.; {Pope}, A.C.; {Quinn}, T.R.; {Rafikov}, R.R.; {Raymond}, S.N.;
  {Richards}, G.T.; {Richmond}, M.W.; {Rix}, H.W.; {Rockosi}, C.M.; {Schaye},
  J.; {Schlegel}, D.J.; {Schneider}, D.P.; {Schroeder}, J.; {Scranton}, R.;
  {Sekiguchi}, M.; {Seljak}, U.; {Sergey}, G.; {Sesar}, B.; {Sheldon}, E.;
  {Shimasaku}, K.; {Siegmund}, W.A.; {Silvestri}, N.M.; {Sinisgalli}, A.J.;
  {Sirko}, E.; {Smith}, J.A.; {Smol{\v c}i{\'c}}, V.; {Snedden}, S.A.;
  {Stebbins}, A.; {Steinhardt}, C.; {Stinson}, G.; {Stoughton}, C.; {Strateva},
  I.V.; {Strauss}, M.A.; {SubbaRao}, M.; {Szalay}, A.S.; {Szapudi}, I.;
  {Szkody}, P.; {Tasca}, L.; {Tegmark}, M.; {Thakar}, A.R.; {Tremonti}, C.;
  {Tucker}, D.L.; {Uomoto}, A.; {Vanden Berk}, D.E.; {Vandenberg}, J.;
  {Vogeley}, M.S.; {Voges}, W.; {Vogt}, N.P.; {Walkowicz}, L.M.; {Weinberg},
  D.H.; {West}, A.A.; {White}, S.D.M.; {Wilhite}, B.C.; {Willman}, B.; {Xu},
  Y.; {Yanny}, B.; {Yarger}, J.; {Yasuda}, N.; {Yip}, C.W.; {Yocum}, D.R.;
  {York}, D.G.; {Zakamska}, N.L.; {Zehavi}, I.; {Zheng}, W.; {Zibetti}, S.;
  {Zucker}, D.B.
\newblock {The First Data Release of the Sloan Digital Sky Survey}.
\newblock {\em AJ} {\bf 2003}, {\em 126},~2081--2086,
  \href{http://xxx.lanl.gov/abs/astro-ph/0305492}{{\normalfont
  [astro-ph/0305492]}}.

\bibitem[{Newberg} \em{et~al.}(2002){Newberg}, {Yanny}, {Rockosi}, {Grebel},
  {Rix}, {Brinkmann}, {Csabai}, {Hennessy}, {Hindsley}, {Ibata}, {Ivezi{\'c}},
  {Lamb}, {Nash}, {Odenkirchen}, {Rave}, {Schneider}, {Smith}, {Stolte}, and
  {York}]{newberg02}
{Newberg}, H.J.; {Yanny}, B.; {Rockosi}, C.; {Grebel}, E.K.; {Rix}, H.W.;
  {Brinkmann}, J.; {Csabai}, I.; {Hennessy}, G.; {Hindsley}, R.B.; {Ibata}, R.;
  {Ivezi{\'c}}, Z.; {Lamb}, D.; {Nash}, E.T.; {Odenkirchen}, M.; {Rave}, H.A.;
  {Schneider}, D.P.; {Smith}, J.A.; {Stolte}, A.; {York}, D.G.
\newblock {The Ghost of Sagittarius and Lumps in the Halo of the Milky Way}.
\newblock {\em ApJ} {\bf 2002}, {\em 569},~245--274,
  \href{http://xxx.lanl.gov/abs/astro-ph/0111095}{{\normalfont
  [astro-ph/0111095]}}.

\bibitem[{Belokurov} \em{et~al.}(2006){Belokurov}, {Zucker}, {Evans},
  {Gilmore}, {Vidrih}, {Bramich}, {Newberg}, {Wyse}, {Irwin}, {Fellhauer},
  {Hewett}, {Walton}, {Wilkinson}, {Cole}, {Yanny}, {Rockosi}, {Beers}, {Bell},
  {Brinkmann}, {Ivezi{\'c}}, and {Lupton}]{belokurov06}
{Belokurov}, V.; {Zucker}, D.B.; {Evans}, N.W.; {Gilmore}, G.; {Vidrih}, S.;
  {Bramich}, D.M.; {Newberg}, H.J.; {Wyse}, R.F.G.; {Irwin}, M.J.; {Fellhauer},
  M.; {Hewett}, P.C.; {Walton}, N.A.; {Wilkinson}, M.I.; {Cole}, N.; {Yanny},
  B.; {Rockosi}, C.M.; {Beers}, T.C.; {Bell}, E.F.; {Brinkmann}, J.;
  {Ivezi{\'c}}, {\v Z}.; {Lupton}, R.
\newblock {The Field of Streams: Sagittarius and Its Siblings}.
\newblock {\em ApJ Lett} {\bf 2006}, {\em 642},~L137--L140,
  \href{http://xxx.lanl.gov/abs/astro-ph/0605025}{{\normalfont
  [astro-ph/0605025]}}.

\bibitem[{Bullock} \em{et~al.}(2001){Bullock}, {Kravtsov}, and
  {Weinberg}]{bullock01}
{Bullock}, J.S.; {Kravtsov}, A.V.; {Weinberg}, D.H.
\newblock {Hierarchical Galaxy Formation and Substructure in the Galaxy's
  Stellar Halo}.
\newblock {\em ApJ} {\bf 2001}, {\em 548},~33--46,
  \href{http://xxx.lanl.gov/abs/astro-ph/0007295}{{\normalfont
  [astro-ph/0007295]}}.

\bibitem[{Bullock} and {Johnston}(2005)]{bullock05}
{Bullock}, J.S.; {Johnston}, K.V.
\newblock {Tracing Galaxy Formation with Stellar Halos. I. Methods}.
\newblock {\em ApJ} {\bf 2005}, {\em 635},~931--949,
  \href{http://xxx.lanl.gov/abs/astro-ph/0506467}{{\normalfont
  [astro-ph/0506467]}}.

\bibitem[{Nikolaev} \em{et~al.}(2000){Nikolaev}, {Weinberg}, {Skrutskie},
  {Cutri}, {Wheelock}, {Gizis}, and {Howard}]{nikolaev00}
{Nikolaev}, S.; {Weinberg}, M.D.; {Skrutskie}, M.F.; {Cutri}, R.M.; {Wheelock},
  S.L.; {Gizis}, J.E.; {Howard}, E.M.
\newblock {A Global Photometric Analysis of 2MASS Calibration Data}.
\newblock {\em AJ} {\bf 2000}, {\em 120},~3340--3350,
  \href{http://xxx.lanl.gov/abs/astro-ph/0008002}{{\normalfont
  [astro-ph/0008002]}}.

\bibitem[{Majewski} \em{et~al.}(2003){Majewski}, {Skrutskie}, {Weinberg}, and
  {Ostheimer}]{majewski03}
{Majewski}, S.R.; {Skrutskie}, M.F.; {Weinberg}, M.D.; {Ostheimer}, J.C.
\newblock {A Two Micron All Sky Survey View of the Sagittarius Dwarf Galaxy. I.
  Morphology of the Sagittarius Core and Tidal Arms}.
\newblock {\em ApJ} {\bf 2003}, {\em 599},~1082--1115,
  \href{http://xxx.lanl.gov/abs/astro-ph/0304198}{{\normalfont
  [astro-ph/0304198]}}.

\bibitem[{Law} and {Majewski}(2010)]{law10}
{Law}, D.R.; {Majewski}, S.R.
\newblock {The Sagittarius Dwarf Galaxy: A Model for Evolution in a Triaxial
  Milky Way Halo}.
\newblock {\em ApJ} {\bf 2010}, {\em 714},~229--254,
  \href{http://xxx.lanl.gov/abs/1003.1132}{{\normalfont [1003.1132]}}.

\bibitem[{Sheffield} \em{et~al.}(2014){Sheffield}, {Johnston}, {Majewski},
  {Damke}, {Richardson}, {Beaton}, and {Rocha-Pinto}]{sheffield14}
{Sheffield}, A.A.; {Johnston}, K.V.; {Majewski}, S.R.; {Damke}, G.;
  {Richardson}, W.; {Beaton}, R.; {Rocha-Pinto}, H.J.
\newblock {Exploring Halo Substructure with Giant Stars. XIV. The Nature of the
  Triangulum-Andromeda Stellar Features}.
\newblock {\em ApJ} {\bf 2014}, {\em 793},~62,
  \href{http://xxx.lanl.gov/abs/1407.4463}{{\normalfont [1407.4463]}}.

\bibitem[{Slater} \em{et~al.}(2014){Slater}, {Bell}, {Schlafly}, {Morganson},
  {Martin}, {Rix}, {Pe{\~n}arrubia}, {Bernard}, {Ferguson}, {Martinez-Delgado},
  {Wyse}, {Burgett}, {Chambers}, {Draper}, {Hodapp}, {Kaiser}, {Magnier},
  {Metcalfe}, {Price}, {Tonry}, {Wainscoat}, and {Waters}]{slater14}
{Slater}, C.T.; {Bell}, E.F.; {Schlafly}, E.F.; {Morganson}, E.; {Martin},
  N.F.; {Rix}, H.W.; {Pe{\~n}arrubia}, J.; {Bernard}, E.J.; {Ferguson}, A.M.N.;
  {Martinez-Delgado}, D.; {Wyse}, R.F.G.; {Burgett}, W.S.; {Chambers}, K.C.;
  {Draper}, P.W.; {Hodapp}, K.W.; {Kaiser}, N.; {Magnier}, E.A.; {Metcalfe},
  N.; {Price}, P.A.; {Tonry}, J.L.; {Wainscoat}, R.J.; {Waters}, C.
\newblock {The Complex Structure of Stars in the Outer Galactic Disk as
  Revealed by Pan-STARRS1}.
\newblock {\em ApJ} {\bf 2014}, {\em 791},~9,
  \href{http://xxx.lanl.gov/abs/1406.6368}{{\normalfont [1406.6368]}}.

\bibitem[{Martin} \em{et~al.}(2014){Martin}, {Ibata}, {Rich}, {Collins},
  {Fardal}, {Irwin}, {Lewis}, {McConnachie}, {Babul}, {Bate}, {Chapman},
  {Conn}, {Crnojevi{\'c}}, {Ferguson}, {Mackey}, {Navarro}, {Pe{\~n}arrubia},
  {Tanvir}, and {Valls-Gabaud}]{martin14}
{Martin}, N.F.; {Ibata}, R.A.; {Rich}, R.M.; {Collins}, M.L.M.; {Fardal}, M.A.;
  {Irwin}, M.J.; {Lewis}, G.F.; {McConnachie}, A.W.; {Babul}, A.; {Bate}, N.F.;
  {Chapman}, S.C.; {Conn}, A.R.; {Crnojevi{\'c}}, D.; {Ferguson}, A.M.N.;
  {Mackey}, A.D.; {Navarro}, J.F.; {Pe{\~n}arrubia}, J.; {Tanvir}, N.T.;
  {Valls-Gabaud}, D.
\newblock {The PAndAS Field of Streams: Stellar Structures in the Milky Way
  Halo toward Andromeda and Triangulum}.
\newblock {\em ApJ} {\bf 2014}, {\em 787},~19,
  \href{http://xxx.lanl.gov/abs/1403.4945}{{\normalfont [1403.4945]}}.

\bibitem[{Deason} \em{et~al.}(2014){Deason}, {Belokurov}, {Hamren}, {Koposov},
  {Gilbert}, {Beaton}, {Dorman}, {Guhathakurta}, {Majewski}, and
  {Cunningham}]{deason14}
{Deason}, A.J.; {Belokurov}, V.; {Hamren}, K.M.; {Koposov}, S.E.; {Gilbert},
  K.M.; {Beaton}, R.L.; {Dorman}, C.E.; {Guhathakurta}, P.; {Majewski}, S.R.;
  {Cunningham}, E.C.
\newblock {TriAnd and its siblings: satellites of satellites in the Milky Way
  halo}.
\newblock {\em MNRAS} {\bf 2014}, {\em 444},~3975--3985,
  \href{http://xxx.lanl.gov/abs/1407.4458}{{\normalfont [1407.4458]}}.

\bibitem[{Robin} \em{et~al.}(1992){Robin}, {Creze}, and {Mohan}]{robin92}
{Robin}, A.C.; {Creze}, M.; {Mohan}, V.
\newblock {The edge of the Galactic disk}.
\newblock {\em ApJ Lett} {\bf 1992}, {\em 400},~L25--L27,
  \href{http://xxx.lanl.gov/abs/astro-ph/9210001}{{\normalfont
  [astro-ph/9210001]}}.

\bibitem[{Morganson} \em{et~al.}(2016){Morganson}, {Conn}, {Rix}, {Bell},
  {Burgett}, {Chambers}, {Dolphin}, {Draper}, {Flewelling}, {Hodapp}, {Kaiser},
  {Magnier}, {Martin}, {Martinez-Delgado}, {Metcalfe}, {Schlafly}, {Slater},
  {Wainscoat}, and {Waters}]{Morganson:2016}
{Morganson}, E.; {Conn}, B.; {Rix}, H.W.; {Bell}, E.F.; {Burgett}, W.S.;
  {Chambers}, K.; {Dolphin}, A.; {Draper}, P.W.; {Flewelling}, H.; {Hodapp},
  K.; {Kaiser}, N.; {Magnier}, E.A.; {Martin}, N.F.; {Martinez-Delgado}, D.;
  {Metcalfe}, N.; {Schlafly}, E.F.; {Slater}, C.T.; {Wainscoat}, R.J.;
  {Waters}, C.Z.
\newblock {Mapping the Monoceros Ring in 3D with Pan-STARRS1}.
\newblock {\em ApJ} {\bf 2016}, {\em 825},~140,
  \href{http://xxx.lanl.gov/abs/1604.07501}{{\normalfont [1604.07501]}}.

\bibitem[{Yanny} \em{et~al.}(2003){Yanny}, {Newberg}, {Grebel}, {Kent},
  {Odenkirchen}, {Rockosi}, {Schlegel}, {Subbarao}, {Brinkmann}, {Fukugita},
  {Ivezic}, {Lamb}, {Schneider}, and {York}]{yanny03}
{Yanny}, B.; {Newberg}, H.J.; {Grebel}, E.K.; {Kent}, S.; {Odenkirchen}, M.;
  {Rockosi}, C.M.; {Schlegel}, D.; {Subbarao}, M.; {Brinkmann}, J.; {Fukugita},
  M.; {Ivezic}, {\v Z}.; {Lamb}, D.Q.; {Schneider}, D.P.; {York}, D.G.
\newblock {A Low-Latitude Halo Stream around the Milky Way}.
\newblock {\em ApJ} {\bf 2003}, {\em 588},~824--841,
  \href{http://xxx.lanl.gov/abs/astro-ph/0301029}{{\normalfont
  [astro-ph/0301029]}}.

\bibitem[{Ibata} \em{et~al.}(2003){Ibata}, {Irwin}, {Lewis}, {Ferguson}, and
  {Tanvir}]{ibata03}
{Ibata}, R.A.; {Irwin}, M.J.; {Lewis}, G.F.; {Ferguson}, A.M.N.; {Tanvir}, N.
\newblock {One ring to encompass them all: a giant stellar structure that
  surrounds the Galaxy}.
\newblock {\em MNRAS} {\bf 2003}, {\em 340},~L21--L27,
  \href{http://xxx.lanl.gov/abs/astro-ph/0301067}{{\normalfont
  [astro-ph/0301067]}}.

\bibitem[{Rocha-Pinto} \em{et~al.}(2003){Rocha-Pinto}, {Majewski}, {Skrutskie},
  and {Crane}]{rochapinto03}
{Rocha-Pinto}, H.J.; {Majewski}, S.R.; {Skrutskie}, M.F.; {Crane}, J.D.
\newblock {Tracing the Galactic Anticenter Stellar Stream with 2MASS M Giants}.
\newblock {\em ApJ Lett} {\bf 2003}, {\em 594},~L115--L118.

\bibitem[{Rocha-Pinto} \em{et~al.}(2004){Rocha-Pinto}, {Majewski}, {Skrutskie},
  {Crane}, and {Patterson}]{rochapinto04}
{Rocha-Pinto}, H.J.; {Majewski}, S.R.; {Skrutskie}, M.F.; {Crane}, J.D.;
  {Patterson}, R.J.
\newblock {Exploring Halo Substructure with Giant Stars: A Diffuse Star Cloud
  or Tidal Debris around the Milky Way in Triangulum-Andromeda}.
\newblock {\em ApJ} {\bf 2004}, {\em 615},~732--737,
  \href{http://xxx.lanl.gov/abs/astro-ph/0405437}{{\normalfont
  [astro-ph/0405437]}}.

\bibitem[{Majewski} \em{et~al.}(2004){Majewski}, {Ostheimer}, {Rocha-Pinto},
  {Patterson}, {Guhathakurta}, and {Reitzel}]{majewski04}
{Majewski}, S.R.; {Ostheimer}, J.C.; {Rocha-Pinto}, H.J.; {Patterson}, R.J.;
  {Guhathakurta}, P.; {Reitzel}, D.
\newblock {Detection of the Main-Sequence Turnoff of a Newly Discovered Milky
  Way Halo Structure in the Triangulum-Andromeda Region}.
\newblock {\em ApJ} {\bf 2004}, {\em 615},~738--743,
  \href{http://xxx.lanl.gov/abs/astro-ph/0406221}{{\normalfont
  [astro-ph/0406221]}}.

\bibitem[{Martin} \em{et~al.}(2007){Martin}, {Ibata}, and {Irwin}]{martin07}
{Martin}, N.F.; {Ibata}, R.A.; {Irwin}, M.
\newblock {Galactic Halo Stellar Structures in the Triangulum-Andromeda
  Region}.
\newblock {\em ApJ Lett} {\bf 2007}, {\em 668},~L123--L126,
  \href{http://xxx.lanl.gov/abs/astro-ph/0703506}{{\normalfont
  [astro-ph/0703506]}}.

\bibitem[{Sharma} and {Johnston}(2009)]{sharma09}
{Sharma}, S.; {Johnston}, K.V.
\newblock {A Group Finding Algorithm for Multidimensional Data Sets}.
\newblock {\em ApJ} {\bf 2009}, {\em 703},~1061--1077.

\bibitem[{Sharma} \em{et~al.}(2010){Sharma}, {Johnston}, {Majewski},
  {Mu{\~n}oz}, {Carlberg}, and {Bullock}]{sharma10}
{Sharma}, S.; {Johnston}, K.V.; {Majewski}, S.R.; {Mu{\~n}oz}, R.R.;
  {Carlberg}, J.K.; {Bullock}, J.
\newblock {Group Finding in the Stellar Halo Using M-giants in the Two Micron
  All Sky Survey: An Extended View of the Pisces Overdensity?}
\newblock {\em ApJ} {\bf 2010}, {\em 722},~750--759,
  \href{http://xxx.lanl.gov/abs/1009.0924}{{\normalfont [1009.0924]}}.

\bibitem[{Pe{\~n}arrubia} \em{et~al.}(2005){Pe{\~n}arrubia},
  {Mart{\'{\i}}nez-Delgado}, {Rix}, {G{\'o}mez-Flechoso}, {Munn}, {Newberg},
  {Bell}, {Yanny}, {Zucker}, and {Grebel}]{penarrubia05}
{Pe{\~n}arrubia}, J.; {Mart{\'{\i}}nez-Delgado}, D.; {Rix}, H.W.;
  {G{\'o}mez-Flechoso}, M.A.; {Munn}, J.; {Newberg}, H.; {Bell}, E.F.; {Yanny},
  B.; {Zucker}, D.; {Grebel}, E.K.
\newblock {A Comprehensive Model for the Monoceros Tidal Stream}.
\newblock {\em ApJ} {\bf 2005}, {\em 626},~128--144,
  \href{http://xxx.lanl.gov/abs/astro-ph/0410448}{{\normalfont
  [astro-ph/0410448]}}.

\bibitem[{Momany} \em{et~al.}(2004){Momany}, {Zaggia}, {Bonifacio}, {Piotto},
  {De Angeli}, {Bedin}, and {Carraro}]{momany04}
{Momany}, Y.; {Zaggia}, S.R.; {Bonifacio}, P.; {Piotto}, G.; {De Angeli}, F.;
  {Bedin}, L.R.; {Carraro}, G.
\newblock {Probing the Canis Major stellar over-density as due to the Galactic
  warp}.
\newblock {\em A\&A} {\bf 2004}, {\em 421},~L29--L32,
  \href{http://xxx.lanl.gov/abs/astro-ph/0405526}{{\normalfont
  [astro-ph/0405526]}}.

\bibitem[{Momany} \em{et~al.}(2006){Momany}, {Zaggia}, {Gilmore}, {Piotto},
  {Carraro}, {Bedin}, and {de Angeli}]{momany06}
{Momany}, Y.; {Zaggia}, S.; {Gilmore}, G.; {Piotto}, G.; {Carraro}, G.;
  {Bedin}, L.R.; {de Angeli}, F.
\newblock {Outer structure of the Galactic warp and flare: explaining the Canis
  Major over-density}.
\newblock {\em A\&A} {\bf 2006}, {\em 451},~515--538,
  \href{http://xxx.lanl.gov/abs/astro-ph/0603385}{{\normalfont
  [astro-ph/0603385]}}.

\bibitem[{Kazantzidis} \em{et~al.}(2008){Kazantzidis}, {Bullock}, {Zentner},
  {Kravtsov}, and {Moustakas}]{kazantzidis08}
{Kazantzidis}, S.; {Bullock}, J.S.; {Zentner}, A.R.; {Kravtsov}, A.V.;
  {Moustakas}, L.A.
\newblock {Cold Dark Matter Substructure and Galactic Disks. I. Morphological
  Signatures of Hierarchical Satellite Accretion}.
\newblock {\em ApJ} {\bf 2008}, {\em 688},~254--276,
  \href{http://xxx.lanl.gov/abs/0708.1949}{{\normalfont [0708.1949]}}.

\bibitem[{Younger} \em{et~al.}(2008){Younger}, {Besla}, {Cox}, {Hernquist},
  {Robertson}, and {Willman}]{younger08}
{Younger}, J.D.; {Besla}, G.; {Cox}, T.J.; {Hernquist}, L.; {Robertson}, B.;
  {Willman}, B.
\newblock {On the Origin of Dynamically Cold Rings around the Milky Way}.
\newblock {\em ApJ Lett} {\bf 2008}, {\em 676},~L21,
  \href{http://xxx.lanl.gov/abs/0802.0872}{{\normalfont [0802.0872]}}.

\bibitem[{Purcell} \em{et~al.}(2011){Purcell}, {Bullock}, {Tollerud}, {Rocha},
  and {Chakrabarti}]{purcell11}
{Purcell}, C.W.; {Bullock}, J.S.; {Tollerud}, E.J.; {Rocha}, M.; {Chakrabarti},
  S.
\newblock {The Sagittarius impact as an architect of spirality and outer rings
  in the Milky Way}.
\newblock {\em Nature} {\bf 2011}, {\em 477},~301--303,
  \href{http://xxx.lanl.gov/abs/1109.2918}{{\normalfont
  [arXiv:astro-ph.GA/1109.2918]}}.

\bibitem[{Xu} \em{et~al.}(2015){Xu}, {Newberg}, {Carlin}, {Liu}, {Deng}, {Li},
  {Sch{\"o}nrich}, and {Yanny}]{xu15}
{Xu}, Y.; {Newberg}, H.J.; {Carlin}, J.L.; {Liu}, C.; {Deng}, L.; {Li}, J.;
  {Sch{\"o}nrich}, R.; {Yanny}, B.
\newblock {Rings and Radial Waves in the Disk of the Milky Way}.
\newblock {\em ApJ} {\bf 2015}, {\em 801},~105,
  \href{http://xxx.lanl.gov/abs/1503.00257}{{\normalfont [1503.00257]}}.

\bibitem[{G{\'o}mez} \em{et~al.}(2016){G{\'o}mez}, {White}, {Marinacci},
  {Slater}, {Grand}, {Springel}, and {Pakmor}]{gomez16}
{G{\'o}mez}, F.A.; {White}, S.D.M.; {Marinacci}, F.; {Slater}, C.T.; {Grand},
  R.J.J.; {Springel}, V.; {Pakmor}, R.
\newblock {A fully cosmological model of a Monoceros-like ring}.
\newblock {\em MNRAS} {\bf 2016}, {\em 456},~2779--2793,
  \href{http://xxx.lanl.gov/abs/1509.08459}{{\normalfont [1509.08459]}}.

\bibitem[{Price-Whelan} \em{et~al.}(2015){Price-Whelan}, {Johnston},
  {Sheffield}, {Laporte}, and {Sesar}]{pricewhelan15}
{Price-Whelan}, A.M.; {Johnston}, K.V.; {Sheffield}, A.A.; {Laporte}, C.F.P.;
  {Sesar}, B.
\newblock {A reinterpretation of the Triangulum-Andromeda stellar clouds: a
  population of halo stars kicked out of the Galactic disc}.
\newblock {\em MNRAS} {\bf 2015}, {\em 452},~676--685,
  \href{http://xxx.lanl.gov/abs/1503.08780}{{\normalfont [1503.08780]}}.

\bibitem[{Koposov} \em{et~al.}(2010){Koposov}, {Rix}, and {Hogg}]{koposov10}
{Koposov}, S.E.; {Rix}, H.W.; {Hogg}, D.W.
\newblock {Constraining the Milky Way Potential with a Six-Dimensional
  Phase-Space Map of the GD-1 Stellar Stream}.
\newblock {\em ApJ} {\bf 2010}, {\em 712},~260--273,
  \href{http://xxx.lanl.gov/abs/0907.1085}{{\normalfont
  [arXiv:astro-ph.GA/0907.1085]}}.

\bibitem[{K{\"u}pper} \em{et~al.}(2015){K{\"u}pper}, {Balbinot}, {Bonaca},
  {Johnston}, {Hogg}, {Kroupa}, and {Santiago}]{kuepper15}
{K{\"u}pper}, A.H.W.; {Balbinot}, E.; {Bonaca}, A.; {Johnston}, K.V.; {Hogg},
  D.W.; {Kroupa}, P.; {Santiago}, B.X.
\newblock {Globular Cluster Streams as Galactic High-Precision
  Scales --- the Poster Child Palomar 5}.
\newblock {\em ApJ} {\bf 2015}, {\em 803},~80,
  \href{http://xxx.lanl.gov/abs/1502.02658}{{\normalfont [1502.02658]}}.

\bibitem[{Bovy} \em{et~al.}(2016){Bovy}, {Bahmanyar}, {Fritz}, and
  {Kallivayalil}]{bovy16}
{Bovy}, J.; {Bahmanyar}, A.; {Fritz}, T.K.; {Kallivayalil}, N.
\newblock {The Shape of the Inner Milky Way Halo from Observations of the Pal 5
  and GD--1 Stellar Streams}.
\newblock {\em ApJ} {\bf 2016}, {\em 833},~31,
  \href{http://xxx.lanl.gov/abs/1609.01298}{{\normalfont [1609.01298]}}.

\bibitem[{Johnston} \em{et~al.}(2008){Johnston}, {Bullock}, {Sharma}, {Font},
  {Robertson}, and {Leitner}]{johnston08}
{Johnston}, K.V.; {Bullock}, J.S.; {Sharma}, S.; {Font}, A.; {Robertson}, B.E.;
  {Leitner}, S.N.
\newblock {Tracing Galaxy Formation with Stellar Halos. II. Relating
  Substructure in Phase and Abundance Space to Accretion Histories}.
\newblock {\em ApJ} {\bf 2008}, {\em 689},~936--957,
  \href{http://xxx.lanl.gov/abs/0807.3911}{{\normalfont [0807.3911]}}.

\bibitem[{Drake} \em{et~al.}(2014){Drake}, {Graham}, {Djorgovski}, {Catelan},
  {Mahabal}, {Torrealba}, {Garc{\'{\i}}a-{\'A}lvarez}, {Donalek}, {Prieto},
  {Williams}, {Larson}, {Christen sen}, {Belokurov}, {Koposov}, {Beshore},
  {Boattini}, {Gibbs}, {Hill}, {Kowalski}, {Johnson}, and {Shelly}]{drake14}
{Drake}, A.J.; {Graham}, M.J.; {Djorgovski}, S.G.; {Catelan}, M.; {Mahabal},
  A.A.; {Torrealba}, G.; {Garc{\'{\i}}a-{\'A}lvarez}, D.; {Donalek}, C.;
  {Prieto}, J.L.; {Williams}, R.; {Larson}, S.; {Christen sen}, E.;
  {Belokurov}, V.; {Koposov}, S.E.; {Beshore}, E.; {Boattini}, A.; {Gibbs}, A.;
  {Hill}, R.; {Kowalski}, R.; {Johnson}, J.; {Shelly}, F.
\newblock {The Catalina Surveys Periodic Variable Star Catalog}.
\newblock {\em ApJ Supp} {\bf 2014}, {\em 213},~9,
  \href{http://xxx.lanl.gov/abs/1405.4290}{{\normalfont
  [arXiv:astro-ph.SR/1405.4290]}}.

\bibitem[{Kirby} \em{et~al.}(2011){Kirby}, {Lanfranchi}, {Simon}, {Cohen}, and
  {Guhathakurta}]{kirby11}
{Kirby}, E.N.; {Lanfranchi}, G.A.; {Simon}, J.D.; {Cohen}, J.G.;
  {Guhathakurta}, P.
\newblock {Multi-element Abundance Measurements from Medium-resolution Spectra.
  III. Metallicity Distributions of Milky Way Dwarf Satellite Galaxies}.
\newblock {\em ApJ} {\bf 2011}, {\em 727},~78,
  \href{http://xxx.lanl.gov/abs/1011.4937}{{\normalfont
  [arXiv:astro-ph.GA/1011.4937]}}.

\bibitem[{Amrose} and {Mckay}(2001)]{amrose01}
{Amrose}, S.; {Mckay}, T.
\newblock {A Calculation of the Mean Local RR Lyrae Space Density Using ROTSE}.
\newblock {\em ApJ Lett} {\bf 2001}, {\em 560},~L151--L154.

\bibitem[{Chou} \em{et~al.}(2010{Natureexlab{a}}){Chou}, {Majewski}, {Cunha},
  {Smith}, {Patterson}, and {Mart{\'{\i}}nez-Delgado}]{chou2010b}
{Chou}, M.Y.; {Majewski}, S.R.; {Cunha}, K.; {Smith}, V.V.; {Patterson}, R.J.;
  {Mart{\'{\i}}nez-Delgado}, D.
\newblock {The Chemical Evolution of the Monoceros Ring/Galactic Anticenter
  Stellar Structure}.
\newblock {\em ApJ Lett} {\bf 2010}, {\em 720},~L5--L10,
  \href{http://xxx.lanl.gov/abs/1007.1056}{{\normalfont [1007.1056]}}.

\bibitem[{Chou} \em{et~al.}(2010{Natureexlab{b}}){Chou}, {Cunha}, {Majewski},
  {Smith}, {Patterson}, {Mart{\'{\i}}nez-Delgado}, and {Geisler}]{chou2010a}
{Chou}, M.Y.; {Cunha}, K.; {Majewski}, S.R.; {Smith}, V.V.; {Patterson}, R.J.;
  {Mart{\'{\i}}nez-Delgado}, D.; {Geisler}, D.
\newblock {A Two Micron All Sky Survey View of the Sagittarius Dwarf Galaxy.
  VI. s-Process and Titanium Abundance Variations Along the Sagittarius
  Stream}.
\newblock {\em ApJ} {\bf 2010}, {\em 708},~1290--1309,
  \href{http://xxx.lanl.gov/abs/0911.4364}{{\normalfont [0911.4364]}}.

\bibitem[{Sbordone} \em{et~al.}(2005){Sbordone}, {Bonifacio}, {Marconi},
  {Zaggia}, and {Buonanno}]{sbordone2005}
{Sbordone}, L.; {Bonifacio}, P.; {Marconi}, G.; {Zaggia}, S.; {Buonanno}, R.
\newblock {UVES observations of the Canis Major overdensity}.
\newblock {\em A\&A} {\bf 2005}, {\em 430},~L13--L16,
  \href{http://xxx.lanl.gov/abs/astro-ph/0412098}{{\normalfont
  [astro-ph/0412098]}}.

\bibitem[Vogt et al.(1994)]{vogt1994} Vogt, S.~S., and 26 colleagues 1994.\ HIRES: the high-resolution echelle spectrometer on the Keck 10-m Telescope.\ Instrumentation in Astronomy VIII 2198, 362. 

\bibitem[{Bergemann}(2011)]{bergemann2011}
{Bergemann}, M.
\newblock {Ionization balance of Ti in the photospheres of the Sun and four
  late-type stars}.
\newblock {\em MNRAS} {\bf 2011}, {\em 413},~2184--2198,
  \href{http://xxx.lanl.gov/abs/1101.0828}{{\normalfont
  [arXiv:astro-ph.SR/1101.0828]}}.

\bibitem[{Bergemann} \em{et~al.}(2012){Bergemann}, {Lind}, {Collet}, {Magic},
  and {Asplund}]{bergemann2012}
{Bergemann}, M.; {Lind}, K.; {Collet}, R.; {Magic}, Z.; {Asplund}, M.
\newblock {Non-LTE line formation of Fe in late-type stars - I. Standard stars
  with 1D and 3D model atmospheres}.
\newblock {\em MNRAS} {\bf 2012}, {\em 427},~27--49,
  \href{http://xxx.lanl.gov/abs/1207.2455}{{\normalfont
  [arXiv:astro-ph.SR/1207.2455]}}.

\bibitem[{Widrow} \em{et~al.}(2012){Widrow}, {Gardner}, {Yanny}, {Dodelson},
  and {Chen}]{widrow12}
{Widrow}, L.M.; {Gardner}, S.; {Yanny}, B.; {Dodelson}, S.; {Chen}, H.Y.
\newblock {Galactoseismology: Discovery of Vertical Waves in the Galactic
  Disk}.
\newblock {\em ApJ Lett} {\bf 2012}, {\em 750},~L41,
  \href{http://xxx.lanl.gov/abs/1203.6861}{{\normalfont [1203.6861]}}.

\bibitem[{Yanny} and {Gardner}(2013)]{yanny13}
{Yanny}, B.; {Gardner}, S.
\newblock {The Stellar Number Density Distribution in the Local Solar
  Neighborhood is North-South Asymmetric}.
\newblock {\em ApJ} {\bf 2013}, {\em 777},~91,
  \href{http://xxx.lanl.gov/abs/1309.2300}{{\normalfont [1309.2300]}}.

\bibitem[{Williams} \em{et~al.}(2013){Williams}, {Steinmetz}, {Binney},
  {Siebert}, {Enke}, {Famaey}, {Minchev}, {de Jong}, {Boeche}, {Freeman},
  {Bienaym{\'e}}, {Bland-Hawthorn}, {Gibson}, {Gilmore}, {Grebel}, {Helmi},
  {Kordopatis}, {Munari}, {Navarro}, {Parker}, {Reid}, {Seabroke}, {Sharma},
  {Siviero}, {Watson}, {Wyse}, and {Zwitter}]{williams13}
{Williams}, M.E.K.; {Steinmetz}, M.; {Binney}, J.; {Siebert}, A.; {Enke}, H.;
  {Famaey}, B.; {Minchev}, I.; {de Jong}, R.S.; {Boeche}, C.; {Freeman}, K.C.;
  {Bienaym{\'e}}, O.; {Bland-Hawthorn}, J.; {Gibson}, B.K.; {Gilmore}, G.F.;
  {Grebel}, E.K.; {Helmi}, A.; {Kordopatis}, G.; {Munari}, U.; {Navarro}, J.F.;
  {Parker}, Q.A.; {Reid}, W.; {Seabroke}, G.M.; {Sharma}, S.; {Siviero}, A.;
  {Watson}, F.G.; {Wyse}, R.F.G.; {Zwitter}, T.
\newblock {The wobbly Galaxy: kinematics north and south with RAVE red-clump
  giants}.
\newblock {\em MNRAS} {\bf 2013}, {\em 436},~101--121,
  \href{http://xxx.lanl.gov/abs/1302.2468}{{\normalfont [1302.2468]}}.

\bibitem[{Cui} \em{et~al.}(2012){Cui}, {Zhao}, {Chu}, {Li}, {Li}, {Zhang},
  {Su}, {Yao}, {Wang}, {Xing}, {Li}, {Zhu}, {Wang}, {Gu}, {Luo}, {Xu}, {Zhang},
  {Liu}, {Zhang}, {Yang}, {Cao}, {Chen}, {Chen}, {Chen}, {Chen}, {Chu}, {Feng},
  {Gong}, {Hou}, {Hu}, {Hu}, {Hu}, {Jia}, {Jiang}, {Jiang}, {Jiang}, {Jin},
  {Li}, {Li}, {Li}, {Liu}, {Liu}, {Lu}, {Mao}, {Men}, {Qi}, {Qi}, {Shi},
  {Tang}, {Tao}, {Wang}, {Wang}, {Wang}, {Wang}, {Wang}, {Wang}, {Wang},
  {Wang}, {Wang}, {Wang}, {Wang}, {Wang}, {Xu}, {Xu}, {Yang}, {Yu}, {Yuan},
  {Yuan}, {Zhai}, {Zhang}, {Zhang}, {Zhang}, {Zhao}, {Zhou}, {Zhou}, {Zhu}, and
  {Zou}]{cui12}
{Cui}, X.Q.; {Zhao}, Y.H.; {Chu}, Y.Q.; {Li}, G.P.; {Li}, Q.; {Zhang}, L.P.;
  {Su}, H.J.; {Yao}, Z.Q.; {Wang}, Y.N.; {Xing}, X.Z.; {Li}, X.N.; {Zhu}, Y.T.;
  {Wang}, G.; {Gu}, B.Z.; {Luo}, A.L.; {Xu}, X.Q.; {Zhang}, Z.C.; {Liu}, G.R.;
  {Zhang}, H.T.; {Yang}, D.H.; {Cao}, S.Y.; {Chen}, H.Y.; {Chen}, J.J.; {Chen},
  K.X.; {Chen}, Y.; {Chu}, J.R.; {Feng}, L.; {Gong}, X.F.; {Hou}, Y.H.; {Hu},
  H.Z.; {Hu}, N.S.; {Hu}, Z.W.; {Jia}, L.; {Jiang}, F.H.; {Jiang}, X.; {Jiang},
  Z.B.; {Jin}, G.; {Li}, A.H.; {Li}, Y.; {Li}, Y.P.; {Liu}, G.Q.; {Liu}, Z.G.;
  {Lu}, W.Z.; {Mao}, Y.D.; {Men}, L.; {Qi}, Y.J.; {Qi}, Z.X.; {Shi}, H.M.;
  {Tang}, Z.H.; {Tao}, Q.S.; {Wang}, D.Q.; {Wang}, D.; {Wang}, G.M.; {Wang},
  H.; {Wang}, J.N.; {Wang}, J.; {Wang}, J.L.; {Wang}, J.P.; {Wang}, L.; {Wang},
  S.Q.; {Wang}, Y.; {Wang}, Y.F.; {Xu}, L.Z.; {Xu}, Y.; {Yang}, S.H.; {Yu}, Y.;
  {Yuan}, H.; {Yuan}, X.Y.; {Zhai}, C.; {Zhang}, J.; {Zhang}, Y.X.; {Zhang},
  Y.; {Zhao}, M.; {Zhou}, F.; {Zhou}, G.H.; {Zhu}, J.; {Zou}, S.C.
\newblock {The Large Sky Area Multi-Object Fiber Spectroscopic Telescope
  (LAMOST)}.
\newblock {\em Research in Astronomy and Astrophysics} {\bf 2012}, {\em
  12},~1197--1242.

\bibitem[{Zhao} \em{et~al.}(2012){Zhao}, {Zhao}, {Chu}, {Jing}, and
  {Deng}]{zhao12}
{Zhao}, G.; {Zhao}, Y.H.; {Chu}, Y.Q.; {Jing}, Y.P.; {Deng}, L.C.
\newblock {LAMOST spectral survey {---} An overview}.
\newblock {\em Research in Astronomy and Astrophysics} {\bf 2012}, {\em
  12},~723--734.

\bibitem[{Carlin} \em{et~al.}(2013){Carlin}, {DeLaunay}, {Newberg}, {Deng},
  {Gole}, {Grabowski}, {Jin}, {Liu}, {Liu}, {Luo}, {Yuan}, {Zhang}, {Zhao}, and
  {Zhao}]{carlin13}
{Carlin}, J.L.; {DeLaunay}, J.; {Newberg}, H.J.; {Deng}, L.; {Gole}, D.;
  {Grabowski}, K.; {Jin}, G.; {Liu}, C.; {Liu}, X.; {Luo}, A.L.; {Yuan}, H.;
  {Zhang}, H.; {Zhao}, G.; {Zhao}, Y.
\newblock {Substructure in Bulk Velocities of Milky Way Disk Stars}.
\newblock {\em ApJ Lett} {\bf 2013}, {\em 777},~L5,
  \href{http://xxx.lanl.gov/abs/1309.6314}{{\normalfont [1309.6314]}}.

\bibitem[{Widrow} \em{et~al.}(2014){Widrow}, {Barber}, {Chequers}, and
  {Cheng}]{widrow14}
{Widrow}, L.M.; {Barber}, J.; {Chequers}, M.H.; {Cheng}, E.
\newblock {Bending and breathing modes of the Galactic disc}.
\newblock {\em MNRAS} {\bf 2014}, {\em 440},~1971--1981,
  \href{http://xxx.lanl.gov/abs/1404.4069}{{\normalfont [1404.4069]}}.

\bibitem[{Kaiser} \em{et~al.}(2010){Kaiser}, {Burgett}, {Chambers}, {Denneau},
  {Heasley}, {Jedicke}, {Magnier}, {Morgan}, {Onaka}, and {Tonry}]{kaiser10}
{Kaiser}, N.; {Burgett}, W.; {Chambers}, K.; {Denneau}, L.; {Heasley}, J.;
  {Jedicke}, R.; {Magnier}, E.; {Morgan}, J.; {Onaka}, P.; {Tonry}, J.
\newblock {The Pan-STARRS wide-field optical/NIR imaging survey}.
\newblock  Ground-based and Airborne Telescopes III,  2010, Vol. 7733, p. 77330E.

\bibitem[{Weinberg} and {Blitz}(2006)]{weinberg06}
{Weinberg}, M.D.; {Blitz}, L.
\newblock {A Magellanic Origin for the Warp of the Galaxy}.
\newblock {\em ApJ Lett} {\bf 2006}, {\em 641},~L33--L36,
  \href{http://xxx.lanl.gov/abs/astro-ph/0601694}{{\normalfont
  [astro-ph/0601694]}}.

\bibitem[{G{\'o}mez} \em{et~al.}(2013){G{\'o}mez}, {Minchev}, {O'Shea},
  {Beers}, {Bullock}, and {Purcell}]{gomez13}
{G{\'o}mez}, F.A.; {Minchev}, I.; {O'Shea}, B.W.; {Beers}, T.C.; {Bullock},
  J.S.; {Purcell}, C.W.
\newblock {Vertical density waves in the Milky Way disc induced by the
  Sagittarius dwarf galaxy}.
\newblock {\em MNRAS} {\bf 2013}, {\em 429},~159--164,
  \href{http://xxx.lanl.gov/abs/1207.3083}{{\normalfont [1207.3083]}}.

\bibitem[{Carlin} \em{et~al.}(2010){Carlin}, {Casetti-Dinescu}, {Grillmair},
  {Majewski}, and {Girard}]{carlin10}
{Carlin}, J.L.; {Casetti-Dinescu}, D.I.; {Grillmair}, C.J.; {Majewski}, S.R.;
  {Girard}, T.M.
\newblock {Kinematics in Kapteyn's Selected Area 76: Orbital Motions Within the
  Highly Substructured Anticenter Stream}.
\newblock {\em ApJ} {\bf 2010}, {\em 725},~2290--2311,
  \href{http://xxx.lanl.gov/abs/1010.5257}{{\normalfont [1010.5257]}}.

\bibitem[{Li} \em{et~al.}(2012){Li}, {Newberg}, {Carlin}, {Deng}, {Newby},
  {Willett}, {Xu}, and {Luo}]{li12}
{Li}, J.; {Newberg}, H.J.; {Carlin}, J.L.; {Deng}, L.; {Newby}, M.; {Willett},
  B.A.; {Xu}, Y.; {Luo}, Z.
\newblock {On Rings and Streams in the Galactic Anti-Center}.
\newblock {\em ApJ} {\bf 2012}, {\em 757},~151,
  \href{http://xxx.lanl.gov/abs/1206.3842}{{\normalfont [1206.3842]}}.

\bibitem[{Ferguson} \em{et~al.}(2002){Ferguson}, {Irwin}, {Ibata}, {Lewis}, and
  {Tanvir}]{ferguson02}
{Ferguson}, A.M.N.; {Irwin}, M.J.; {Ibata}, R.A.; {Lewis}, G.F.; {Tanvir}, N.R.
\newblock {Evidence for Stellar Substructure in the Halo and Outer Disk of
  M31}.
\newblock {\em AJ} {\bf 2002}, {\em 124},~1452--1463,
  \href{http://xxx.lanl.gov/abs/astro-ph/0205530}{{\normalfont
  [astro-ph/0205530]}}.

\bibitem[{Ibata} \em{et~al.}(2005){Ibata}, {Chapman}, {Ferguson}, {Lewis},
  {Irwin}, and {Tanvir}]{ibata05}
{Ibata}, R.; {Chapman}, S.; {Ferguson}, A.M.N.; {Lewis}, G.; {Irwin}, M.;
  {Tanvir}, N.
\newblock {On the Accretion Origin of a Vast Extended Stellar Disk around the
  Andromeda Galaxy}.
\newblock {\em ApJ} {\bf 2005}, {\em 634},~287--313,
  \href{http://xxx.lanl.gov/abs/astro-ph/0504164}{{\normalfont
  [astro-ph/0504164]}}.

\bibitem[{Toth} and {Ostriker}(1992)]{toth92}
{Toth}, G.; {Ostriker}, J.P.
\newblock {Galactic disks, infall, and the global value of Omega}.
\newblock {\em ApJ} {\bf 1992}, {\em 389},~5--26.

\bibitem[{Quinn} \em{et~al.}(1993){Quinn}, {Hernquist}, and
  {Fullagar}]{quinn93}
{Quinn}, P.J.; {Hernquist}, L.; {Fullagar}, D.P.
\newblock {Heating of galactic disks by mergers}.
\newblock {\em ApJ} {\bf 1993}, {\em 403},~74--93.

\bibitem[{Walker} \em{et~al.}(1996){Walker}, {Mihos}, and
  {Hernquist}]{walker96}
{Walker}, I.R.; {Mihos}, J.C.; {Hernquist}, L.
\newblock {Quantifying the Fragility of Galactic Disks in Minor Mergers}.
\newblock {\em ApJ} {\bf 1996}, {\em 460},~121,
  \href{http://xxx.lanl.gov/abs/astro-ph/9510052}{{\normalfont
  [astro-ph/9510052]}}.

\bibitem[{Font} \em{et~al.}(2001){Font}, {Navarro}, {Stadel}, and
  {Quinn}]{font01}
{Font}, A.S.; {Navarro}, J.F.; {Stadel}, J.; {Quinn}, T.
\newblock {Halo Substructure and Disk Heating in a {$\Lambda$} Cold Dark Matter
  Universe}.
\newblock {\em ApJ Lett} {\bf 2001}, {\em 563},~L1--L4,
  \href{http://xxx.lanl.gov/abs/astro-ph/0106268}{{\normalfont
  [astro-ph/0106268]}}.

\bibitem[{Ardi} \em{et~al.}(2003){Ardi}, {Tsuchiya}, and {Burkert}]{ardi03}
{Ardi}, E.; {Tsuchiya}, T.; {Burkert}, A.
\newblock {Constraints of the Clumpiness of Dark Matter Halos through Heating
  of the Disk Galaxies}.
\newblock {\em ApJ} {\bf 2003}, {\em 596},~204--215,
  \href{http://xxx.lanl.gov/abs/astro-ph/0206026}{{\normalfont
  [astro-ph/0206026]}}.

\bibitem[{Benson} \em{et~al.}(2004){Benson}, {Lacey}, {Frenk}, {Baugh}, and
  {Cole}]{benson04}
{Benson}, A.J.; {Lacey}, C.G.; {Frenk}, C.S.; {Baugh}, C.M.; {Cole}, S.
\newblock {Heating of galactic discs by infalling satellites}.
\newblock {\em MNRAS} {\bf 2004}, {\em 351},~1215--1236,
  \href{http://xxx.lanl.gov/abs/astro-ph/0307298}{{\normalfont
  [astro-ph/0307298]}}.

\bibitem[{Stewart} \em{et~al.}(2008){Stewart}, {Bullock}, {Wechsler}, {Maller},
  and {Zentner}]{stewart08}
{Stewart}, K.R.; {Bullock}, J.S.; {Wechsler}, R.H.; {Maller}, A.H.; {Zentner},
  A.R.
\newblock {Merger Histories of Galaxy Halos and Implications for Disk
  Survival}.
\newblock {\em ApJ} {\bf 2008}, {\em 683},~597--610,
  \href{http://xxx.lanl.gov/abs/0711.5027}{{\normalfont [0711.5027]}}.

\bibitem[{Hopkins} \em{et~al.}(2008){Hopkins}, {Hernquist}, {Cox}, {Younger},
  and {Besla}]{hopkins08}
{Hopkins}, P.F.; {Hernquist}, L.; {Cox}, T.J.; {Younger}, J.D.; {Besla}, G.
\newblock {The Radical Consequences of Realistic Satellite Orbits for the
  Heating and Implied Merger Histories of Galactic Disks}.
\newblock {\em ApJ} {\bf 2008}, {\em 688},~757--769,
  \href{http://xxx.lanl.gov/abs/0806.2861}{{\normalfont [0806.2861]}}.

\bibitem[{Villalobos} and {Helmi}(2008)]{villalobos08}
{Villalobos}, {\'A}.; {Helmi}, A.
\newblock {Simulations of minor mergers - I. General properties of thick
  discs}.
\newblock {\em MNRAS} {\bf 2008}, {\em 391},~1806--1827.

\bibitem[{Purcell} \em{et~al.}(2009){Purcell}, {Kazantzidis}, and
  {Bullock}]{purcell09}
{Purcell}, C.W.; {Kazantzidis}, S.; {Bullock}, J.S.
\newblock {The Destruction of Thin Stellar Disks Via Cosmologically Common
  Satellite Accretion Events}.
\newblock {\em ApJ Lett} {\bf 2009}, {\em 694},~L98--L102,
  \href{http://xxx.lanl.gov/abs/0810.2785}{{\normalfont [0810.2785]}}.

\bibitem[{Kazantzidis} \em{et~al.}(2009){Kazantzidis}, {Zentner}, {Kravtsov},
  {Bullock}, and {Debattista}]{kazantzidis09}
{Kazantzidis}, S.; {Zentner}, A.R.; {Kravtsov}, A.V.; {Bullock}, J.S.;
  {Debattista}, V.P.
\newblock {Cold Dark Matter Substructure and Galactic Disks. II. Dynamical
  Effects of Hierarchical Satellite Accretion}.
\newblock {\em ApJ} {\bf 2009}, {\em 700},~1896--1920,
  \href{http://xxx.lanl.gov/abs/0902.1983}{{\normalfont
  [arXiv:astro-ph.GA/0902.1983]}}.

\bibitem[{Sachdeva} and {Saha}(2016)]{sachdeva16}
{Sachdeva}, S.; {Saha}, K.
\newblock {Survival of Pure Disk Galaxies over the Last 8 Billion Years}.
\newblock {\em ApJ Lett} {\bf 2016}, {\em 820},~L4,
  \href{http://xxx.lanl.gov/abs/1602.08942}{{\normalfont [1602.08942]}}.

\bibitem[{Moetazedian} and {Just}(2016)]{moetazedian16}
{Moetazedian}, R.; {Just}, A.
\newblock {Impact of cosmological satellites on the vertical heating of the
  Milky Way disc}.
\newblock {\em MNRAS} {\bf 2016}, {\em 459},~2905--2924,
  \href{http://xxx.lanl.gov/abs/1508.03580}{{\normalfont [1508.03580]}}.

\bibitem[{Abadi} \em{et~al.}(2006){Abadi}, {Navarro}, and {Steinmetz}]{abadi06}
{Abadi}, M.G.; {Navarro}, J.F.; {Steinmetz}, M.
\newblock {Stars beyond galaxies: the origin of extended luminous haloes around
  galaxies}.
\newblock {\em MNRAS} {\bf 2006}, {\em 365},~747--758,
  \href{http://xxx.lanl.gov/abs/astro-ph/0506659}{{\normalfont
  [astro-ph/0506659]}}.

\bibitem[{Zolotov} \em{et~al.}(2009){Zolotov}, {Willman}, {Brooks},
  {Governato}, {Brook}, {Hogg}, {Quinn}, and {Stinson}]{zolotov09}
{Zolotov}, A.; {Willman}, B.; {Brooks}, A.M.; {Governato}, F.; {Brook}, C.B.;
  {Hogg}, D.W.; {Quinn}, T.; {Stinson}, G.
\newblock {The Dual Origin of Stellar Halos}.
\newblock {\em ApJ} {\bf 2009}, {\em 702},~1058--1067,
  \href{http://xxx.lanl.gov/abs/0904.3333}{{\normalfont
  [arXiv:astro-ph.GA/0904.3333]}}.

\bibitem[{Zolotov} \em{et~al.}(2010){Zolotov}, {Willman}, {Brooks},
  {Governato}, {Hogg}, {Shen}, and {Wadsley}]{zolotov10}
{Zolotov}, A.; {Willman}, B.; {Brooks}, A.M.; {Governato}, F.; {Hogg}, D.W.;
  {Shen}, S.; {Wadsley}, J.
\newblock {The Dual Origin of Stellar Halos. II. Chemical Abundances as Tracers
  of Formation History}.
\newblock {\em ApJ} {\bf 2010}, {\em 721},~738--743,
  \href{http://xxx.lanl.gov/abs/1004.3789}{{\normalfont [1004.3789]}}.

\bibitem[{Font} \em{et~al.}(2011){Font}, {McCarthy}, {Crain}, {Theuns},
  {Schaye}, {Wiersma}, and {Dalla Vecchia}]{font11}
{Font}, A.S.; {McCarthy}, I.G.; {Crain}, R.A.; {Theuns}, T.; {Schaye}, J.;
  {Wiersma}, R.P.C.; {Dalla Vecchia}, C.
\newblock {Cosmological simulations of the formation of the stellar haloes
  around disc galaxies}.
\newblock {\em MNRAS} {\bf 2011}, {\em 416},~2802--2820,
  \href{http://xxx.lanl.gov/abs/1102.2526}{{\normalfont [1102.2526]}}.

\bibitem[{McCarthy} \em{et~al.}(2012){McCarthy}, {Font}, {Crain}, {Deason},
  {Schaye}, and {Theuns}]{mccarthy12}
{McCarthy}, I.G.; {Font}, A.S.; {Crain}, R.A.; {Deason}, A.J.; {Schaye}, J.;
  {Theuns}, T.
\newblock {Global structure and kinematics of stellar haloes in cosmological
  hydrodynamic simulations}.
\newblock {\em MNRAS} {\bf 2012}, {\em 420},~2245--2262,
  \href{http://xxx.lanl.gov/abs/1111.1747}{{\normalfont
  [arXiv:astro-ph.GA/1111.1747]}}.

\bibitem[{Tissera} \em{et~al.}(2013){Tissera}, {Scannapieco}, {Beers}, and
  {Carollo}]{tissera13}
{Tissera}, P.B.; {Scannapieco}, C.; {Beers}, T.C.; {Carollo}, D.
\newblock {Stellar haloes of simulated Milky-Way-like galaxies: chemical and
  kinematic properties}.
\newblock {\em MNRAS} {\bf 2013}, {\em 432},~3391--3400,
  \href{http://xxx.lanl.gov/abs/1301.1301}{{\normalfont [1301.1301]}}.

\bibitem[{Tissera} \em{et~al.}(2014){Tissera}, {Beers}, {Carollo}, and
  {Scannapieco}]{tissera14}
{Tissera}, P.B.; {Beers}, T.C.; {Carollo}, D.; {Scannapieco}, C.
\newblock {Stellar haloes in Milky Way mass galaxies: from the inner to the
  outer haloes}.
\newblock {\em MNRAS} {\bf 2014}, {\em 439},~3128--3138,
  \href{http://xxx.lanl.gov/abs/1309.3609}{{\normalfont
  [arXiv:astro-ph.CO/1309.3609]}}.

\bibitem[{Pillepich} \em{et~al.}(2015){Pillepich}, {Madau}, and
  {Mayer}]{pillepich15}
{Pillepich}, A.; {Madau}, P.; {Mayer}, L.
\newblock {Building Late-type Spiral Galaxies by In-situ and Ex-situ Star
  Formation}.
\newblock {\em ApJ} {\bf 2015}, {\em 799},~184,
  \href{http://xxx.lanl.gov/abs/1407.7855}{{\normalfont [1407.7855]}}.

\bibitem[{Cooper} \em{et~al.}(2015){Cooper}, {Parry}, {Lowing}, {Cole}, and
  {Frenk}]{cooper15}
{Cooper}, A.P.; {Parry}, O.H.; {Lowing}, B.; {Cole}, S.; {Frenk}, C.
\newblock {Formation of in situ stellar haloes in Milky Way-mass galaxies}.
\newblock {\em MNRAS} {\bf 2015}, {\em 454},~3185--3199,
  \href{http://xxx.lanl.gov/abs/1501.04630}{{\normalfont [1501.04630]}}.

\bibitem[{Carollo} \em{et~al.}(2007){Carollo}, {Beers}, {Lee}, {Chiba},
  {Norris}, {Wilhelm}, {Sivarani}, {Marsteller}, {Munn}, {Bailer-Jones},
  {Fiorentin}, and {York}]{carollo07}
{Carollo}, D.; {Beers}, T.C.; {Lee}, Y.S.; {Chiba}, M.; {Norris}, J.E.;
  {Wilhelm}, R.; {Sivarani}, T.; {Marsteller}, B.; {Munn}, J.A.;
  {Bailer-Jones}, C.A.L.; {Fiorentin}, P.R.; {York}, D.G.
\newblock {Two stellar components in the halo of the Milky Way}.
\newblock {\em Nature} {\bf 2007}, {\em 450},~1020--1025,
  \href{http://xxx.lanl.gov/abs/0706.3005}{{\normalfont [0706.3005]}}.

\bibitem[{Deason} \em{et~al.}(2013){Deason}, {Belokurov}, {Evans}, and
  {Johnston}]{deason13}
{Deason}, A.J.; {Belokurov}, V.; {Evans}, N.W.; {Johnston}, K.V.
\newblock {Broken and Unbroken: The Milky Way and M31 Stellar Halos}.
\newblock {\em ApJ} {\bf 2013}, {\em 763},~113,
  \href{http://xxx.lanl.gov/abs/1210.4929}{{\normalfont [1210.4929]}}.

\bibitem[{Dorman} \em{et~al.}(2013){Dorman}, {Widrow}, {Guhathakurta}, {Seth},
  {Foreman-Mackey}, {Bell}, {Dalcanton}, {Gilbert}, {Skillman}, and
  {Williams}]{dorman13}
{Dorman}, C.E.; {Widrow}, L.M.; {Guhathakurta}, P.; {Seth}, A.C.;
  {Foreman-Mackey}, D.; {Bell}, E.F.; {Dalcanton}, J.J.; {Gilbert}, K.M.;
  {Skillman}, E.D.; {Williams}, B.F.
\newblock {A New Approach to Detailed Structural Decomposition from the SPLASH
  and PHAT Surveys: Kicked-up Disk Stars in the Andromeda Galaxy?}
\newblock {\em ApJ} {\bf 2013}, {\em 779},~103,
  \href{http://xxx.lanl.gov/abs/1310.4179}{{\normalfont [1310.4179]}}.

\bibitem[{Hawkins} \em{et~al.}(2015){Hawkins}, {Kordopatis}, {Gilmore},
  {Masseron}, {Wyse}, {Ruchti}, {Bienaym{\'e}}, {Bland-Hawthorn}, {Boeche},
  {Freeman}, {Gibson}, {Grebel}, {Helmi}, {Kunder}, {Munari}, {Navarro},
  {Parker}, {Reid}, {Scholz}, {Seabroke}, {Siebert}, {Steinmetz}, {Watson}, and
  {Zwitter}]{hawkins15}
{Hawkins}, K.; {Kordopatis}, G.; {Gilmore}, G.; {Masseron}, T.; {Wyse}, R.F.G.;
  {Ruchti}, G.; {Bienaym{\'e}}, O.; {Bland-Hawthorn}, J.; {Boeche}, C.;
  {Freeman}, K.; {Gibson}, B.K.; {Grebel}, E.K.; {Helmi}, A.; {Kunder}, A.;
  {Munari}, U.; {Navarro}, J.F.; {Parker}, Q.A.; {Reid}, W.A.; {Scholz}, R.D.;
  {Seabroke}, G.; {Siebert}, A.; {Steinmetz}, M.; {Watson}, F.; {Zwitter}, T.
\newblock {Characterizing the high-velocity stars of RAVE: the discovery of a
  metal-rich halo star born in the Galactic disc}.
\newblock {\em MNRAS} {\bf 2015}, {\em 447},~2046--2058,
  \href{http://xxx.lanl.gov/abs/1412.1484}{{\normalfont [1412.1484]}}.

\bibitem[{Bonaca} \em{et~al.}(2017){Bonaca}, {Conroy}, {Wetzel}, {Hopkins}, and
  {Keres}]{bonaca17}
{Bonaca}, A.; {Conroy}, C.; {Wetzel}, A.; {Hopkins}, P.F.; {Keres}, D.
\newblock {Gaia reveals a metal-rich in-situ component of the local stellar
  halo}.
\newblock {\em ArXiv e-prints} {\bf 2017},
  \href{http://xxx.lanl.gov/abs/1704.05463}{{\normalfont [1704.05463]}}.

\bibitem[{Chakrabarti} and {Blitz}(2009)]{chakrabarti09}
{Chakrabarti}, S.; {Blitz}, L.
\newblock {Tidal imprints of a dark subhalo on the outskirts of the Milky Way}.
\newblock {\em MNRAS} {\bf 2009}, {\em 399},~L118--L122,
  \href{http://xxx.lanl.gov/abs/0812.0821}{{\normalfont [0812.0821]}}.

\bibitem[{Chakrabarti} \em{et~al.}(2011){Chakrabarti}, {Bigiel}, {Chang}, and
  {Blitz}]{chakrabarti11b}
{Chakrabarti}, S.; {Bigiel}, F.; {Chang}, P.; {Blitz}, L.
\newblock {Finding Dwarf Galaxies from Their Tidal Imprints}.
\newblock {\em ApJ} {\bf 2011}, {\em 743},~35,
  \href{http://xxx.lanl.gov/abs/1101.0815}{{\normalfont [1101.0815]}}.

\bibitem[{Chang} and {Chakrabarti}(2011)]{chang11}
{Chang}, P.; {Chakrabarti}, S.
\newblock {Dark subhaloes and disturbances in extended H I discs}.
\newblock {\em MNRAS} {\bf 2011}, {\em 416},~618--628,
  \href{http://xxx.lanl.gov/abs/1102.3436}{{\normalfont [1102.3436]}}.

\bibitem[Duc et al.(2015)]{duc15} Duc, P.-A., and 26 colleagues 2015.\ The ATLAS$^{3D}$ project - XXIX. The new look of early-type galaxies and surrounding fields disclosed by extremely deep optical images.\ Monthly Notices of the Royal Astronomical Society 446, 120-143. 

\bibitem[Mart{\'{\i}}nez-Delgado et al.(2010)]{delgado10} Mart{\'{\i}}nez-Delgado, D., and 14 colleagues 2010.\ Stellar Tidal Streams in Spiral Galaxies of the Local Volume: A Pilot Survey with Modest Aperture Telescopes.\ The Astronomical Journal 140, 962-967. 

\bibitem[van Dokkum et al.(2014)]{vandokkum14} van Dokkum, P.~G., Abraham, R., Merritt, A.\ 2014.\ First Results from the Dragonfly Telephoto Array: The Apparent Lack of a Stellar Halo in the Massive Spiral Galaxy M101.\ The Astrophysical Journal 782, L24. 

\bibitem[Spergel et al.(2013)]{spergel13} Spergel, D., and 45 colleagues 2013.\ WFIRST-2.4: What Every Astronomer Should Know.\ ArXiv e-prints arXiv:1305.5425. 

\bibitem[Ivezic et al.(2008)]{ivezic08} Ivezic, Z., and 26 colleagues 2008.\ Large Synoptic Survey Telescope: From Science Drivers To Reference Design.\ Serbian Astronomical Journal 176, 1-13. 

\bibitem[Monachesi et al.(2013)]{monachesi13} Monachesi, A., Bell, E.~F., Radburn-Smith, D.~J., Vlaji{\'c}, M., de Jong, R.~S., Bailin, J., Dalcanton, J.~J., Holwerda, B.~W., Streich, D.\ 2013.\ Testing Galaxy Formation Models with the GHOSTS Survey: The Color Profile of M81's Stellar Halo.\ The Astrophysical Journal 766, 106. 

\bibitem[Crnojevi{\'c} et al.(2016)]{crnojevi16} Crnojevi{\'c}, D., Sand, D.~J., Spekkens, K., Caldwell, N., Guhathakurta, P., McLeod, B., Seth, A., Simon, J.~D., Strader, J., Toloba, E.\ 2016.\ The Extended Halo of Centaurus A: Uncovering Satellites, Streams, and Substructures.\ The Astrophysical Journal 823, 19. 

\bibitem[D'Onghia et al.(2016)]{donghia16} D'Onghia, E., Madau, P., Vera-Ciro, C., Quillen, A., Hernquist, L.\ 2016.\ Excitation of Coupled Stellar Motions in the Galactic Disk by Orbiting Satellites.\ The Astrophysical Journal 823, 4. 

\bibitem[Crane et al.(2003)]{crane03} Crane, J.~D., Majewski, S.~R., Rocha-Pinto, H.~J., Frinchaboy, P.~M., Skrutskie, M.~F., Law, D.~R.\ 2003.\ Exploring Halo Substructure with Giant Stars: Spectroscopy of Stars in the Galactic Anticenter Stellar Structure.\ The Astrophysical Journal 594, L119-L122. 

\bibitem[Chou et al.(2011)]{chou11} Chou, M.-Y., Majewski, S.~R., Cunha, K., Smith, V.~V., Patterson, R.~J., Mart{\'{\i}}nez-Delgado, D.\ 2011.\ First Chemical Analysis of Stars in the Triangulum--Andromeda Star Cloud.\ The Astrophysical Journal 731, L30. 

\bibitem[Sheffield et al.(2012)]{sheffield12} Sheffield, A.~A., and 12 colleagues 2012.\ Identifying Contributions to the Stellar Halo from Accreted, Kicked-out, and In Situ Populations.\ The Astrophysical Journal 761, 161. 

\bibitem[Velazquez and White(1999)]{velazquez99} Velazquez, H., White, S.~D.~M.\ 1999.\ Sinking satellites and the heating of galaxy discs.\ Monthly Notices of the Royal Astronomical Society 304, 254-270. 

%\bibitem[Kazantzidis et al.(2008)]{kazantzidis08} Kazantzidis, S., Bullock, J.~S., Zentner, A.~R., Kravtsov, A.~V., Moustakas, L.~A.\ 2008.\ Cold Dark Matter Substructure and Galactic Disks. I. Morphological Signatures of Hierarchical Satellite Accretion.\ The Astrophysical Journal 688, 254-276. 

\bibitem[Deng et al.(2012)]{deng12} Deng, L.-C., and 25 colleagues 2012.\ LAMOST Experiment for Galactic Understanding and Exploration (LEGUE) {---} The survey's science plan.\ Research in Astronomy and Astrophysics 12, 735-754. 

\bibitem[Laporte et al.(2016)]{laporte16} Laporte, C.~F.~P., G{\'o}mez, F.~A., Besla, G., Johnston, K.~V., Garavito-Camargo, N.\ 2016.\ Response of the Milky Way's disc to the Large Magellanic Cloud in a first infall scenario.\ ArXiv e-prints arXiv:1608.04743. 

\bibitem[Li et al.(2017)]{li17} Li, T.~S., and 10 colleagues 2017.\ Exploring Halo Substructure with Giant Stars. XV. Discovery of a Connection between the Monoceros Ring and the Triangulum-Andromeda Overdensity?.\ ArXiv e-prints arXiv:1703.05384. 

\end{thebibliography}

%\begin{thebibliography}{999}
% Reference 1
%\bibitem[Author1(year)]{ref-journal}
%Author1, T. The title of the cited article. {\em Journal Abbreviation} {\bf 2008}, {\em 10}, 142-149.
% Reference 2
%\%bibitem[Author2(year)]{ref-book}
%Author2, L. The title of the cited contribution. In {\em The Book Title}; Editor1, F., Editor2, A., Eds.; %Publishing House: City, Country, 2007; pp. 32-58.
%\end{thebibliography}


%%%%%%%%%%%%%%%%%%%%%%%%%%%%%%%%%%%%%%%%%%
\end{document}

% APW TODO:
% We note that (1) these structures likely continue towards lower Galactic
% latitudes ($b < 20^\circ$) but are incomplete due to Galactic extinction near
% the midplane, and (2) each structure in reality has a longitude-dependent
% distance that is ignored in the quoted distance ranges above.
% Many models have been proposed to explain the formation and existence of these
% substructures.
